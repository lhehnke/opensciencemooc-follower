\documentclass[]{article}
\usepackage{lmodern}
\usepackage{amssymb,amsmath}
\usepackage{ifxetex,ifluatex}
\usepackage{fixltx2e} % provides \textsubscript
\ifnum 0\ifxetex 1\fi\ifluatex 1\fi=0 % if pdftex
  \usepackage[T1]{fontenc}
  \usepackage[utf8]{inputenc}
\else % if luatex or xelatex
  \ifxetex
    \usepackage{mathspec}
  \else
    \usepackage{fontspec}
  \fi
  \defaultfontfeatures{Ligatures=TeX,Scale=MatchLowercase}
\fi
% use upquote if available, for straight quotes in verbatim environments
\IfFileExists{upquote.sty}{\usepackage{upquote}}{}
% use microtype if available
\IfFileExists{microtype.sty}{%
\usepackage{microtype}
\UseMicrotypeSet[protrusion]{basicmath} % disable protrusion for tt fonts
}{}
\usepackage[margin=1in]{geometry}
\usepackage{hyperref}
\hypersetup{unicode=true,
            pdftitle={Analyzing the Open Science MOOC Twitter community},
            pdfauthor={Lisa Hehnke; dataplanes.org \textbar{} @DataPlanes},
            pdfborder={0 0 0},
            breaklinks=true}
\urlstyle{same}  % don't use monospace font for urls
\usepackage{color}
\usepackage{fancyvrb}
\newcommand{\VerbBar}{|}
\newcommand{\VERB}{\Verb[commandchars=\\\{\}]}
\DefineVerbatimEnvironment{Highlighting}{Verbatim}{commandchars=\\\{\}}
% Add ',fontsize=\small' for more characters per line
\usepackage{framed}
\definecolor{shadecolor}{RGB}{248,248,248}
\newenvironment{Shaded}{\begin{snugshade}}{\end{snugshade}}
\newcommand{\KeywordTok}[1]{\textcolor[rgb]{0.13,0.29,0.53}{\textbf{#1}}}
\newcommand{\DataTypeTok}[1]{\textcolor[rgb]{0.13,0.29,0.53}{#1}}
\newcommand{\DecValTok}[1]{\textcolor[rgb]{0.00,0.00,0.81}{#1}}
\newcommand{\BaseNTok}[1]{\textcolor[rgb]{0.00,0.00,0.81}{#1}}
\newcommand{\FloatTok}[1]{\textcolor[rgb]{0.00,0.00,0.81}{#1}}
\newcommand{\ConstantTok}[1]{\textcolor[rgb]{0.00,0.00,0.00}{#1}}
\newcommand{\CharTok}[1]{\textcolor[rgb]{0.31,0.60,0.02}{#1}}
\newcommand{\SpecialCharTok}[1]{\textcolor[rgb]{0.00,0.00,0.00}{#1}}
\newcommand{\StringTok}[1]{\textcolor[rgb]{0.31,0.60,0.02}{#1}}
\newcommand{\VerbatimStringTok}[1]{\textcolor[rgb]{0.31,0.60,0.02}{#1}}
\newcommand{\SpecialStringTok}[1]{\textcolor[rgb]{0.31,0.60,0.02}{#1}}
\newcommand{\ImportTok}[1]{#1}
\newcommand{\CommentTok}[1]{\textcolor[rgb]{0.56,0.35,0.01}{\textit{#1}}}
\newcommand{\DocumentationTok}[1]{\textcolor[rgb]{0.56,0.35,0.01}{\textbf{\textit{#1}}}}
\newcommand{\AnnotationTok}[1]{\textcolor[rgb]{0.56,0.35,0.01}{\textbf{\textit{#1}}}}
\newcommand{\CommentVarTok}[1]{\textcolor[rgb]{0.56,0.35,0.01}{\textbf{\textit{#1}}}}
\newcommand{\OtherTok}[1]{\textcolor[rgb]{0.56,0.35,0.01}{#1}}
\newcommand{\FunctionTok}[1]{\textcolor[rgb]{0.00,0.00,0.00}{#1}}
\newcommand{\VariableTok}[1]{\textcolor[rgb]{0.00,0.00,0.00}{#1}}
\newcommand{\ControlFlowTok}[1]{\textcolor[rgb]{0.13,0.29,0.53}{\textbf{#1}}}
\newcommand{\OperatorTok}[1]{\textcolor[rgb]{0.81,0.36,0.00}{\textbf{#1}}}
\newcommand{\BuiltInTok}[1]{#1}
\newcommand{\ExtensionTok}[1]{#1}
\newcommand{\PreprocessorTok}[1]{\textcolor[rgb]{0.56,0.35,0.01}{\textit{#1}}}
\newcommand{\AttributeTok}[1]{\textcolor[rgb]{0.77,0.63,0.00}{#1}}
\newcommand{\RegionMarkerTok}[1]{#1}
\newcommand{\InformationTok}[1]{\textcolor[rgb]{0.56,0.35,0.01}{\textbf{\textit{#1}}}}
\newcommand{\WarningTok}[1]{\textcolor[rgb]{0.56,0.35,0.01}{\textbf{\textit{#1}}}}
\newcommand{\AlertTok}[1]{\textcolor[rgb]{0.94,0.16,0.16}{#1}}
\newcommand{\ErrorTok}[1]{\textcolor[rgb]{0.64,0.00,0.00}{\textbf{#1}}}
\newcommand{\NormalTok}[1]{#1}
\usepackage{graphicx,grffile}
\makeatletter
\def\maxwidth{\ifdim\Gin@nat@width>\linewidth\linewidth\else\Gin@nat@width\fi}
\def\maxheight{\ifdim\Gin@nat@height>\textheight\textheight\else\Gin@nat@height\fi}
\makeatother
% Scale images if necessary, so that they will not overflow the page
% margins by default, and it is still possible to overwrite the defaults
% using explicit options in \includegraphics[width, height, ...]{}
\setkeys{Gin}{width=\maxwidth,height=\maxheight,keepaspectratio}
\IfFileExists{parskip.sty}{%
\usepackage{parskip}
}{% else
\setlength{\parindent}{0pt}
\setlength{\parskip}{6pt plus 2pt minus 1pt}
}
\setlength{\emergencystretch}{3em}  % prevent overfull lines
\providecommand{\tightlist}{%
  \setlength{\itemsep}{0pt}\setlength{\parskip}{0pt}}
\setcounter{secnumdepth}{0}
% Redefines (sub)paragraphs to behave more like sections
\ifx\paragraph\undefined\else
\let\oldparagraph\paragraph
\renewcommand{\paragraph}[1]{\oldparagraph{#1}\mbox{}}
\fi
\ifx\subparagraph\undefined\else
\let\oldsubparagraph\subparagraph
\renewcommand{\subparagraph}[1]{\oldsubparagraph{#1}\mbox{}}
\fi

%%% Use protect on footnotes to avoid problems with footnotes in titles
\let\rmarkdownfootnote\footnote%
\def\footnote{\protect\rmarkdownfootnote}

%%% Change title format to be more compact
\usepackage{titling}

% Create subtitle command for use in maketitle
\providecommand{\subtitle}[1]{
  \posttitle{
    \begin{center}\large#1\end{center}
    }
}

\setlength{\droptitle}{-2em}

  \title{Analyzing the Open Science MOOC Twitter community}
    \pretitle{\vspace{\droptitle}\centering\huge}
  \posttitle{\par}
    \author{Lisa Hehnke \\ dataplanes.org \textbar{} @DataPlanes}
    \preauthor{\centering\large\emph}
  \postauthor{\par}
      \predate{\centering\large\emph}
  \postdate{\par}
    \date{01 June, 2019}


\begin{document}
\maketitle

\subsection{Introduction}\label{introduction}

As of lately, several friends of mine tried to convince me to resume
writing and this was when I remembered that I apparently once thought so
as well when creating this blog. One of those friends happened to be
\href{http://fossilsandshit.com/}{Jon Tennant}, dinosaur whisperer from
Brexit land and founder of the \href{https://opensciencemooc.eu/}{Open
Science MOOC}.

In case you have never heard of it, the Open Science MOOC is a free
massive open online course hosted on
\href{https://eliademy.com/opensciencemooc}{Eliademy}. It aims to equip
students and researchers with the skills they need to excel in a modern
research environment based on Open Science - that is, the broad adoption
of good scientific practices as a fundamental and essential part of the
research process. The Open MOOC brings together the efforts and
resources of hundreds of researchers and practitioners who have all
dedicated their time and experience to create a welcoming and supporting
community platform.

Along with almost 700 \href{https://openmooc-ers.slack.com/}{Slack}
members, we are currently working on a new module on
\href{https://github.com/OpenScienceMOOC/Module-6-Open-Access-to-Research-Papers}{Open
Access}. While the other Open MOOC-ers are busy collecting resources and
setting learning outcomes, I wanted to contribute with what I can do
best (other than
\href{https://twitter.com/DataPlanes/status/1131308854258610176}{building
dinosaur kits for children}): Analyzing them. Because analyzing people
is what you do when you like them, right? Is that just me? OK, fair
enough.

After realizing that a) I could not get my hands on our Slack analytics
due to the free plan and b) being inspired by Shirin Glander's awesome
\href{https://shiring.github.io/text_analysis/2017/06/28/twitter_post}{blog
post} on characterizing her own Twitter followers, I decided to collect
data on all friends and followers of the Open Science MOOC's official
\href{https://twitter.com/opensciencemooc}{Twitter account}.

In this blog post, I am now going to analyze this data in order to
answer the following questions:

\begin{enumerate}
\def\labelenumi{\arabic{enumi})}
\tightlist
\item
  Is there an overlap between the Open Science MOOC's Twitter followers
  and friends? Who should be immediately unfollowed for not following
  back?*
\item
  Where are the Open Science MOOC's twitter followers based? Is there
  any evidence for a geographically concentrated Open Science Twitter
  bubble?
\item
  What about diversity among the followers? Does the Open Science MOOC
  keep its promise of being an inclusive and diverse platform?
\item
  Who are the most influential and active followers? Could they
  potentially start a revolution, take over the Twitter community, and
  throw Jon off the Open Science throne?
\item
  What do the followers state in their own profile descriptions? Which
  institutions are they affiliated with? Which opinions do they express?
  And last but certainly not least: Do their own texts say anything
  meaningful about the interests and research activities of the Open
  Science Twitter MOOC-ers (spoiler alert: yes)?
\end{enumerate}

\emph{*Just kidding, of course I skipped this part. But make sure to
follow the Open Science MOOC, one never knows what the future might
bring\ldots{} Just kidding again.}

\subsection{Collecting Twitter data}\label{collecting-twitter-data}

As the old saying goes, every analysis needs its data. While I generally
prefer Mike Kearney's \texttt{rtweet} over \texttt{twitteR} for scraping
Twitter data these days, I used the latter for collecting data on all of
the N = 5764 followers and N = 4924 friends, i.e.~followed accounts, of
the Open Science MOOC. The data for this analysis was last retrieved on
May 29, 2019.

And while \texttt{ggplot2} - especially when combined with
\texttt{ggthemes} - comes with beautiful themes by default, I most often
cannot resist to create a custom theme for each project. With the colour
scheme below, I tried to match the visual appearance of Shirin's
follower analysis. And yes, it would have been easier, albeit less fun,
if I had followed her example and simply used \texttt{tidyquant}.

\begin{Shaded}
\begin{Highlighting}[]
\CommentTok{# Set directory}
\NormalTok{MAIN_DIR <-}\StringTok{ }\NormalTok{rprojroot}\OperatorTok{::}\KeywordTok{find_rstudio_root_file}\NormalTok{()}

\CommentTok{# Install and load packages using pacman}
\ControlFlowTok{if}\NormalTok{ (}\OperatorTok{!}\KeywordTok{require}\NormalTok{(}\StringTok{"pacman"}\NormalTok{)) }\KeywordTok{install.packages}\NormalTok{(}\StringTok{"pacman"}\NormalTok{)}
\KeywordTok{library}\NormalTok{(pacman)}

\KeywordTok{p_load}\NormalTok{(genderizeR, ggraph, igraph, leaflet, magrittr, maps, reshape2, SnowballC, tidytext, }
\NormalTok{       tidyverse, tmaptools, twitteR, wordcloud, wordcloud2)}
\end{Highlighting}
\end{Shaded}

\begin{verbatim}
## 
##   There is a binary version available but the source version is
##   later:
##         binary source needs_compilation
## openssl    1.3    1.4              TRUE
## 
##   Binaries will be installed
## package 'sys' successfully unpacked and MD5 sums checked
## package 'askpass' successfully unpacked and MD5 sums checked
## package 'curl' successfully unpacked and MD5 sums checked
## package 'openssl' successfully unpacked and MD5 sums checked
## package 'R6' successfully unpacked and MD5 sums checked
## package 'NLP' successfully unpacked and MD5 sums checked
## package 'slam' successfully unpacked and MD5 sums checked
## package 'xml2' successfully unpacked and MD5 sums checked
## package 'BH' successfully unpacked and MD5 sums checked
## package 'httr' successfully unpacked and MD5 sums checked
## package 'tm' successfully unpacked and MD5 sums checked
## package 'data.table' successfully unpacked and MD5 sums checked
## package 'genderizeR' successfully unpacked and MD5 sums checked
## 
## The downloaded binary packages are in
##  C:\Users\jonte\AppData\Local\Temp\RtmpArkdPY\downloaded_packages
## 
##   There is a binary version available but the source version is
##   later:
##        binary source needs_compilation
## tibble  2.1.1  2.1.2              TRUE
## 
##   Binaries will be installed
## package 'zeallot' successfully unpacked and MD5 sums checked
## package 'utf8' successfully unpacked and MD5 sums checked
## package 'vctrs' successfully unpacked and MD5 sums checked
## package 'cli' successfully unpacked and MD5 sums checked
## package 'crayon' successfully unpacked and MD5 sums checked
## package 'fansi' successfully unpacked and MD5 sums checked
## package 'pillar' successfully unpacked and MD5 sums checked
## package 'purrr' successfully unpacked and MD5 sums checked
## package 'farver' successfully unpacked and MD5 sums checked
## package 'colorspace' successfully unpacked and MD5 sums checked
## package 'lazyeval' successfully unpacked and MD5 sums checked
## package 'reshape2' successfully unpacked and MD5 sums checked
## package 'rlang' successfully unpacked and MD5 sums checked
## package 'tibble' successfully unpacked and MD5 sums checked
## package 'viridisLite' successfully unpacked and MD5 sums checked
## package 'withr' successfully unpacked and MD5 sums checked
## package 'assertthat' successfully unpacked and MD5 sums checked
## package 'pkgconfig' successfully unpacked and MD5 sums checked
## package 'tidyselect' successfully unpacked and MD5 sums checked
## package 'plogr' successfully unpacked and MD5 sums checked
## package 'tweenr' successfully unpacked and MD5 sums checked
## package 'polyclip' successfully unpacked and MD5 sums checked
## package 'RcppEigen' successfully unpacked and MD5 sums checked
## package 'labeling' successfully unpacked and MD5 sums checked
## package 'munsell' successfully unpacked and MD5 sums checked
## package 'RColorBrewer' successfully unpacked and MD5 sums checked
## package 'gridExtra' successfully unpacked and MD5 sums checked
## package 'ggplot2' successfully unpacked and MD5 sums checked
## package 'dplyr' successfully unpacked and MD5 sums checked
## package 'plyr' successfully unpacked and MD5 sums checked
## package 'ggforce' successfully unpacked and MD5 sums checked
## package 'igraph' successfully unpacked and MD5 sums checked
## package 'scales' successfully unpacked and MD5 sums checked
## package 'gtable' successfully unpacked and MD5 sums checked
## package 'ggrepel' successfully unpacked and MD5 sums checked
## package 'viridis' successfully unpacked and MD5 sums checked
## package 'ggraph' successfully unpacked and MD5 sums checked
## 
## The downloaded binary packages are in
##  C:\Users\jonte\AppData\Local\Temp\RtmpArkdPY\downloaded_packages
## package 'httpuv' successfully unpacked and MD5 sums checked
## package 'xtable' successfully unpacked and MD5 sums checked
## package 'sourcetools' successfully unpacked and MD5 sums checked
## package 'later' successfully unpacked and MD5 sums checked
## package 'promises' successfully unpacked and MD5 sums checked
## package 'shiny' successfully unpacked and MD5 sums checked
## package 'crosstalk' successfully unpacked and MD5 sums checked
## package 'htmlwidgets' successfully unpacked and MD5 sums checked
## package 'png' successfully unpacked and MD5 sums checked
## package 'raster' successfully unpacked and MD5 sums checked
## package 'sp' successfully unpacked and MD5 sums checked
## package 'leaflet' successfully unpacked and MD5 sums checked
## 
## The downloaded binary packages are in
##  C:\Users\jonte\AppData\Local\Temp\RtmpArkdPY\downloaded_packages
## package 'maps' successfully unpacked and MD5 sums checked
## 
## The downloaded binary packages are in
##  C:\Users\jonte\AppData\Local\Temp\RtmpArkdPY\downloaded_packages
## package 'SnowballC' successfully unpacked and MD5 sums checked
## 
## The downloaded binary packages are in
##  C:\Users\jonte\AppData\Local\Temp\RtmpArkdPY\downloaded_packages
## package 'generics' successfully unpacked and MD5 sums checked
## package 'tidyr' successfully unpacked and MD5 sums checked
## package 'ISOcodes' successfully unpacked and MD5 sums checked
## package 'hunspell' successfully unpacked and MD5 sums checked
## package 'broom' successfully unpacked and MD5 sums checked
## package 'tokenizers' successfully unpacked and MD5 sums checked
## package 'janeaustenr' successfully unpacked and MD5 sums checked
## package 'stopwords' successfully unpacked and MD5 sums checked
## package 'tidytext' successfully unpacked and MD5 sums checked
## 
## The downloaded binary packages are in
##  C:\Users\jonte\AppData\Local\Temp\RtmpArkdPY\downloaded_packages
## package 'ps' successfully unpacked and MD5 sums checked
## package 'rematch' successfully unpacked and MD5 sums checked
## package 'prettyunits' successfully unpacked and MD5 sums checked
## package 'processx' successfully unpacked and MD5 sums checked
## package 'DBI' successfully unpacked and MD5 sums checked
## package 'ellipsis' successfully unpacked and MD5 sums checked
## package 'clipr' successfully unpacked and MD5 sums checked
## package 'cellranger' successfully unpacked and MD5 sums checked
## package 'progress' successfully unpacked and MD5 sums checked
## package 'callr' successfully unpacked and MD5 sums checked
## package 'fs' successfully unpacked and MD5 sums checked
## package 'whisker' successfully unpacked and MD5 sums checked
## package 'selectr' successfully unpacked and MD5 sums checked
## package 'dbplyr' successfully unpacked and MD5 sums checked
## package 'forcats' successfully unpacked and MD5 sums checked
## package 'haven' successfully unpacked and MD5 sums checked
## package 'hms' successfully unpacked and MD5 sums checked
## package 'lubridate' successfully unpacked and MD5 sums checked
## package 'modelr' successfully unpacked and MD5 sums checked
## package 'readr' successfully unpacked and MD5 sums checked
## package 'readxl' successfully unpacked and MD5 sums checked
## package 'reprex' successfully unpacked and MD5 sums checked
## package 'rstudioapi' successfully unpacked and MD5 sums checked
## package 'rvest' successfully unpacked and MD5 sums checked
## package 'tidyverse' successfully unpacked and MD5 sums checked
## 
## The downloaded binary packages are in
##  C:\Users\jonte\AppData\Local\Temp\RtmpArkdPY\downloaded_packages
## 
##   There is a binary version available but the source version is
##   later:
##       binary source needs_compilation
## rgdal  1.4-3  1.4-4              TRUE
## 
##   Binaries will be installed
## package 'e1071' successfully unpacked and MD5 sums checked
## package 'sf' successfully unpacked and MD5 sums checked
## package 'lwgeom' successfully unpacked and MD5 sums checked
## package 'units' successfully unpacked and MD5 sums checked
## package 'rgdal' successfully unpacked and MD5 sums checked
## package 'rgeos' successfully unpacked and MD5 sums checked
## package 'classInt' successfully unpacked and MD5 sums checked
## package 'dichromat' successfully unpacked and MD5 sums checked
## package 'XML' successfully unpacked and MD5 sums checked
## package 'tmaptools' successfully unpacked and MD5 sums checked
## 
## The downloaded binary packages are in
##  C:\Users\jonte\AppData\Local\Temp\RtmpArkdPY\downloaded_packages
## package 'bit' successfully unpacked and MD5 sums checked
## package 'bit64' successfully unpacked and MD5 sums checked
## package 'rjson' successfully unpacked and MD5 sums checked
## package 'twitteR' successfully unpacked and MD5 sums checked
## 
## The downloaded binary packages are in
##  C:\Users\jonte\AppData\Local\Temp\RtmpArkdPY\downloaded_packages
## package 'wordcloud' successfully unpacked and MD5 sums checked
## 
## The downloaded binary packages are in
##  C:\Users\jonte\AppData\Local\Temp\RtmpArkdPY\downloaded_packages
## package 'wordcloud2' successfully unpacked and MD5 sums checked
## 
## The downloaded binary packages are in
##  C:\Users\jonte\AppData\Local\Temp\RtmpArkdPY\downloaded_packages
\end{verbatim}

\begin{Shaded}
\begin{Highlighting}[]
\CommentTok{# Get information on Open Science MOOC}
\NormalTok{osci_mooc <-}\StringTok{ }\KeywordTok{getUser}\NormalTok{(}\StringTok{"opensciencemooc"}\NormalTok{)}

\CommentTok{# Retrieve followers' profile data}
\NormalTok{osci_followers <-}\StringTok{ }\NormalTok{osci_mooc}\OperatorTok{$}\KeywordTok{getFollowers}\NormalTok{()}
\NormalTok{followers_df <-}\StringTok{ }\KeywordTok{twListToDF}\NormalTok{(osci_followers)}

\CommentTok{# Retrieve friends' profile data}
\NormalTok{osci_friends <-}\StringTok{ }\NormalTok{osci_mooc}\OperatorTok{$}\KeywordTok{getFriends}\NormalTok{()}
\NormalTok{friends_df <-}\StringTok{ }\KeywordTok{twListToDF}\NormalTok{(osci_friends)}

\CommentTok{# Export data}
\KeywordTok{saveRDS}\NormalTok{(followers_df, }\StringTok{"opensciencemooc_followers.rds"}\NormalTok{)}
\KeywordTok{saveRDS}\NormalTok{(friends_df, }\StringTok{"opensciencemooc_friends.rds"}\NormalTok{)}
\end{Highlighting}
\end{Shaded}

\begin{Shaded}
\begin{Highlighting}[]
\CommentTok{# Import Twitter data}
\NormalTok{data_df <-}\StringTok{ }\KeywordTok{readRDS}\NormalTok{(}\DataTypeTok{file =} \KeywordTok{paste}\NormalTok{(MAIN_DIR,}\StringTok{"opensciencemooc_followers.rds"}\NormalTok{, }\DataTypeTok{sep =} \StringTok{"/"}\NormalTok{))}
\end{Highlighting}
\end{Shaded}

\begin{Shaded}
\begin{Highlighting}[]
\NormalTok{viz_theme <-}\StringTok{ }\KeywordTok{theme}\NormalTok{(}
  \DataTypeTok{axis.line =} \KeywordTok{element_line}\NormalTok{(}\DataTypeTok{colour =} \StringTok{"#2c3e50"}\NormalTok{),}
  \DataTypeTok{axis.text =} \KeywordTok{element_text}\NormalTok{(}\DataTypeTok{colour =} \StringTok{"#2c3e50"}\NormalTok{),}
  \DataTypeTok{axis.ticks =} \KeywordTok{element_line}\NormalTok{(}\DataTypeTok{colour =} \StringTok{"#85a0bc"}\NormalTok{),}
  \DataTypeTok{legend.key =} \KeywordTok{element_rect}\NormalTok{(}\DataTypeTok{colour =} \StringTok{"transparent"}\NormalTok{, }\DataTypeTok{fill =} \StringTok{"white"}\NormalTok{), }
  \DataTypeTok{panel.border =} \KeywordTok{element_blank}\NormalTok{(),}
  \DataTypeTok{panel.background =} \KeywordTok{element_blank}\NormalTok{(),}
  \DataTypeTok{panel.grid.major =} \KeywordTok{element_blank}\NormalTok{(),}
  \DataTypeTok{panel.grid.minor =} \KeywordTok{element_blank}\NormalTok{(),}
  \DataTypeTok{plot.caption =} \KeywordTok{element_text}\NormalTok{(}\DataTypeTok{colour =} \StringTok{"#3e5871"}\NormalTok{),}
  \DataTypeTok{strip.background =} \KeywordTok{element_rect}\NormalTok{(}\DataTypeTok{colour =} \StringTok{"#2c3e50"}\NormalTok{, }\DataTypeTok{fill =} \StringTok{"white"}\NormalTok{),}
  \DataTypeTok{strip.text =} \KeywordTok{element_text}\NormalTok{(}\DataTypeTok{size =} \KeywordTok{rel}\NormalTok{(}\DecValTok{1}\NormalTok{)),}
  \DataTypeTok{text =} \KeywordTok{element_text}\NormalTok{(}\DataTypeTok{colour =} \StringTok{"#2c3e50"}\NormalTok{))}
\end{Highlighting}
\end{Shaded}

\subsection{Friend or follow?}\label{friend-or-follow}

For starters, we can get a simple overview of the total number of
friends, followers, and accounts who are friends as well as followers.
As can be easily derived from the publicly available statistics on
Twitter, the Open Science MOOC account obviously has more followers than
friends. Interestingly though, there is quite a large number of accounts
the MOOC follows but that do not follow back in return. Yet, over 2000
out of N = 8638 accounts in total are both friends and followers at the
same time.

\begin{Shaded}
\begin{Highlighting}[]
\CommentTok{# Import data on friends and merge}
\NormalTok{friends_df <-}\StringTok{ }\KeywordTok{readRDS}\NormalTok{(}\DataTypeTok{file =} \KeywordTok{paste}\NormalTok{(MAIN_DIR,}\StringTok{"opensciencemooc_friends.rds"}\NormalTok{, }\DataTypeTok{sep =} \StringTok{"/"}\NormalTok{))}

\NormalTok{relations_df <-}\StringTok{ }\KeywordTok{rbind}\NormalTok{(}\KeywordTok{mutate}\NormalTok{(data_df, }\DataTypeTok{relation =} \KeywordTok{ifelse}\NormalTok{(screenName }\OperatorTok\StringTok{ }\NormalTok{friends_df}\OperatorTok{$}\NormalTok{screenName, }\StringTok{"both"}\NormalTok{, }\StringTok{"follower"}\NormalTok{)),}
                              \KeywordTok{mutate}\NormalTok{(friends_df, }\DataTypeTok{relation =} \KeywordTok{ifelse}\NormalTok{(screenName }\OperatorTok\StringTok{ }\NormalTok{data_df}\OperatorTok{$}\NormalTok{screenName, }\StringTok{"both"}\NormalTok{, }\StringTok{"friend"}\NormalTok{))) }\OperatorTok\StringTok{ }
\StringTok{  }\KeywordTok{distinct}\NormalTok{()}

\CommentTok{# Plot relations}
\NormalTok{relations_df }\OperatorTok\StringTok{ }
\StringTok{  }\KeywordTok{ggplot}\NormalTok{(}\DataTypeTok{mapping =} \KeywordTok{aes}\NormalTok{(}\DataTypeTok{x =}\NormalTok{ relation, }\DataTypeTok{fill =}\NormalTok{ relation)) }\OperatorTok{+}
\StringTok{  }\KeywordTok{scale_fill_manual}\NormalTok{(}\StringTok{" "}\NormalTok{, }\DataTypeTok{values =} \KeywordTok{c}\NormalTok{(}\StringTok{"#6c7a89"}\NormalTok{, }\StringTok{"#2C3E50"}\NormalTok{, }\StringTok{"#22313f"}\NormalTok{)) }\OperatorTok{+}
\StringTok{  }\KeywordTok{scale_x_discrete}\NormalTok{(}\DataTypeTok{labels =} \KeywordTok{c}\NormalTok{(}\StringTok{"Friend & follower"}\NormalTok{, }\StringTok{"Follower"}\NormalTok{, }\StringTok{"Friend"}\NormalTok{)) }\OperatorTok{+}
\StringTok{  }\NormalTok{viz_theme }\OperatorTok{+}\StringTok{ }\KeywordTok{ylab}\NormalTok{(}\StringTok{""}\NormalTok{) }\OperatorTok{+}\StringTok{ }\KeywordTok{xlab}\NormalTok{(}\StringTok{""}\NormalTok{) }\OperatorTok{+}
\StringTok{  }\KeywordTok{geom_bar}\NormalTok{(}\DataTypeTok{alpha =} \FloatTok{0.8}\NormalTok{, }\DataTypeTok{width =} \FloatTok{0.5}\NormalTok{) }\OperatorTok{+}\StringTok{ }\KeywordTok{ylim}\NormalTok{(}\DecValTok{0}\NormalTok{, }\DecValTok{4000}\NormalTok{) }\OperatorTok{+}
\StringTok{  }\KeywordTok{ggtitle}\NormalTok{(}\DataTypeTok{label =} \StringTok{"Count of Open Science MOOC Twitter followers & friends"}\NormalTok{, }\DataTypeTok{subtitle =} \StringTok{" "}\NormalTok{) }\OperatorTok{+}\StringTok{ }
\StringTok{  }\KeywordTok{theme}\NormalTok{(}\DataTypeTok{legend.position =} \StringTok{"none"}\NormalTok{)}
\end{Highlighting}
\end{Shaded}

\begin{center}\includegraphics{OpenScienceMOOC-follower-analysis_files/figure-latex/friends-followers-1} \end{center}

\subsection{Locating the Twitter
bubble}\label{locating-the-twitter-bubble}

After thois first overview of the Open MOOC-ers Twitter community, let
us dive deeper into the whereabouts of the followers. For this purpose,
we can use the locations users provide in their Twitter profiles. Not
all of the followers did this, but the available information proves to
be sufficient for a reasonably accurate overview.

Prior to mapping the locations, however, the unstructured text data
needed some basic cleaning by removing mentions, hashtags, and strings
either consisting of only one character or containing numbers. In
addition, I removed all punctuation except for commas, since locations,
if specified correctly (sorry to disappoint you, Jon, \emph{Jurassic
Park} is not a valid location yet), are displayed in the \emph{city,
country} format. Lastly, I trimmed both leading and trailing whitespace
and converted the preprocessed locations to lower case.

\begin{Shaded}
\begin{Highlighting}[]
\CommentTok{# Clean location}
\NormalTok{## Note: Code removes locations consisting of non-ASCII characters.}
\NormalTok{data_df }\OperatorTok
\StringTok{  }\KeywordTok{mutate}\NormalTok{(}\DataTypeTok{location_clean =} \KeywordTok{gsub}\NormalTok{(}\StringTok{"#}\CharTok{\textbackslash{}\textbackslash{}}\StringTok{S+"}\NormalTok{, }\StringTok{""}\NormalTok{, location)) }\OperatorTok\StringTok{ }\CommentTok{# remove hashtags}
\StringTok{  }\KeywordTok{mutate}\NormalTok{(}\DataTypeTok{location_clean =} \KeywordTok{gsub}\NormalTok{(}\StringTok{"@}\CharTok{\textbackslash{}\textbackslash{}}\StringTok{S+"}\NormalTok{, }\StringTok{""}\NormalTok{, location_clean)) }\OperatorTok\StringTok{ }\CommentTok{# remove mentions}
\StringTok{  }\KeywordTok{mutate}\NormalTok{(}\DataTypeTok{location_clean =} \KeywordTok{gsub}\NormalTok{(}\StringTok{"[^[:alnum:][:space:]}\CharTok{\textbackslash{}\textbackslash{}}\StringTok{,]"}\NormalTok{, }\StringTok{""}\NormalTok{, location_clean)) }\OperatorTok\StringTok{ }\CommentTok{# remove punctuation except for commas}
\StringTok{  }\KeywordTok{mutate}\NormalTok{(}\DataTypeTok{location_clean =} \KeywordTok{trimws}\NormalTok{(}\KeywordTok{gsub}\NormalTok{(}\StringTok{"}\CharTok{\textbackslash{}\textbackslash{}}\StringTok{w*[0-9]+}\CharTok{\textbackslash{}\textbackslash{}}\StringTok{w*}\CharTok{\textbackslash{}\textbackslash{}}\StringTok{s*"}\NormalTok{, }\StringTok{""}\NormalTok{, location_clean))) }\OperatorTok\StringTok{ }\CommentTok{# remove words containing numbers}
\StringTok{  }\KeywordTok{mutate}\NormalTok{(}\DataTypeTok{location_clean =} \KeywordTok{gsub}\NormalTok{(}\StringTok{" *}\CharTok{\textbackslash{}\textbackslash{}}\StringTok{b[[:alpha:]]\{1\}}\CharTok{\textbackslash{}\textbackslash{}}\StringTok{b *"}\NormalTok{, }\StringTok{" "}\NormalTok{, location_clean)) }\OperatorTok\StringTok{ }\CommentTok{# remove one letter words}
\StringTok{  }\KeywordTok{mutate}\NormalTok{(}\DataTypeTok{location_clean =} \KeywordTok{str_trim}\NormalTok{(location_clean, }\DataTypeTok{side =} \StringTok{"both"}\NormalTok{)) }\OperatorTok\StringTok{ }\CommentTok{# remove whitespace}
\StringTok{  }\KeywordTok{mutate}\NormalTok{(}\DataTypeTok{location_clean =} \KeywordTok{gsub}\NormalTok{(}\StringTok{"^[[:punct:]].*"}\NormalTok{, }\StringTok{""}\NormalTok{, location_clean)) }\OperatorTok\StringTok{ }\CommentTok{# remove strings starting with comma(s)}
\StringTok{  }\KeywordTok{mutate}\NormalTok{(}\DataTypeTok{location_clean =} \KeywordTok{gsub}\NormalTok{(}\StringTok{"^$"}\NormalTok{, }\OtherTok{NA}\NormalTok{, }\KeywordTok{trimws}\NormalTok{(location_clean))) }\OperatorTok\StringTok{ }\CommentTok{# replace blank cells with NA}
\StringTok{  }\KeywordTok{mutate}\NormalTok{(}\DataTypeTok{city_clean =} \KeywordTok{str_extract}\NormalTok{(location_clean, }\StringTok{"[^,]+"}\NormalTok{)) }\OperatorTok\StringTok{ }\CommentTok{# extract string before first comma}
\StringTok{  }\KeywordTok{mutate}\NormalTok{(}\DataTypeTok{city_clean =} \KeywordTok{tolower}\NormalTok{(city_clean))}
\end{Highlighting}
\end{Shaded}

The following plot shows the top locations the Open Science MOOC-ers
claimed to be from, after extracting the name of the city from the
string and removing unspecific locations like \emph{europe} or
\emph{uk}. Among the remaining locations are multiple European capital
cities, including London, Berlin, Paris, Amsterdam, and Stockholm as
well as large US-American and Canadian cities such as New York,
Washington, Boston, Toronto or Montréal. Also on the list is Barcelona,
a city know for its far-reaching
\href{https://opendata-ajuntament.barcelona.cat/en/open-data-bcn}{open
data policy}. However, almost all of the frequently mentioned locations
are located in either Western Europe or North America, with Sydney and
Melbourne being the Australian exceptions.

\begin{Shaded}
\begin{Highlighting}[]
\CommentTok{# Remove unspecific locations}
\StringTok{"%notin%"}\NormalTok{ <-}\StringTok{ }\KeywordTok{Negate}\NormalTok{(}\StringTok{"%in%"}\NormalTok{)}
\NormalTok{locations_rm <-}\StringTok{ }\KeywordTok{c}\NormalTok{(}\StringTok{"uk"}\NormalTok{, }\StringTok{"europe"}\NormalTok{) }

\CommentTok{# Plot followers' top locations}
\NormalTok{data_df }\OperatorTok
\StringTok{  }\KeywordTok{filter}\NormalTok{(}\OperatorTok{!}\KeywordTok{is.na}\NormalTok{(city_clean)) }\OperatorTok
\StringTok{  }\KeywordTok{filter}\NormalTok{(city_clean }\OperatorTok\StringTok{ }\NormalTok{locations_rm) }\OperatorTok
\StringTok{  }\KeywordTok{count}\NormalTok{(city_clean, }\DataTypeTok{sort =} \OtherTok{TRUE}\NormalTok{) }\OperatorTok
\StringTok{  }\KeywordTok{mutate}\NormalTok{(}\DataTypeTok{city_clean =} \KeywordTok{reorder}\NormalTok{(city_clean, n)) }\OperatorTok
\StringTok{  }\KeywordTok{top_n}\NormalTok{(}\DecValTok{20}\NormalTok{, n) }\OperatorTok
\StringTok{  }\KeywordTok{ggplot}\NormalTok{(}\KeywordTok{aes}\NormalTok{(city_clean, n)) }\OperatorTok{+}
\StringTok{  }\KeywordTok{geom_bar}\NormalTok{(}\DataTypeTok{stat =} \StringTok{"identity"}\NormalTok{, }\DataTypeTok{width =} \FloatTok{0.5}\NormalTok{, }\DataTypeTok{alpha =} \FloatTok{0.8}\NormalTok{, }\DataTypeTok{color =} \StringTok{"#2c3e50"}\NormalTok{, }\DataTypeTok{fill =} \StringTok{"#2c3e50"}\NormalTok{) }\OperatorTok{+}
\StringTok{  }\KeywordTok{xlab}\NormalTok{(}\StringTok{""}\NormalTok{) }\OperatorTok{+}\StringTok{ }\KeywordTok{ylab}\NormalTok{(}\StringTok{""}\NormalTok{) }\OperatorTok{+}\StringTok{ }\KeywordTok{ggtitle}\NormalTok{(}\StringTok{"Top locations of Open Science Twitter MOOC-ers"}\NormalTok{, }\DataTypeTok{subtitle =} \StringTok{" "}\NormalTok{) }\OperatorTok{+}
\StringTok{  }\KeywordTok{coord_flip}\NormalTok{() }\OperatorTok{+}\StringTok{ }\NormalTok{viz_theme}
\end{Highlighting}
\end{Shaded}

\begin{center}\includegraphics{OpenScienceMOOC-follower-analysis_files/figure-latex/locations-1} \end{center}

To investigate this interesting finding further, I geocoded all
available locations using the
\href{https://wiki.openstreetmap.org/wiki/Nominatim}{OpenStreetMap
Nominatim} API. The \texttt{geocode\_OSM()} function of the
\texttt{tmaptools} package is one of the available options for geocoding
a location to its coordinates; alternatives are \texttt{ggmap} or
\texttt{opencage}. When setting \texttt{as.data.frame\ =\ TRUE},
\texttt{geocode\_OSM()} returns a data frame containing the latitude and
longitude of each location. After obtaining the results, I removed both
unspecific (e.g.~administrative boundaries) and overly specific
locations (e.g.~localities or amenities) by filtering all observations
that match the \texttt{type\ ==\ city} condition.

Fun fact: \texttt{NA}, which indicates missing values in R, apparently
translates to \emph{Norge, Ytterbyvegen, Namsos, Trøndelag, 7810, Norge}
when geocoding it with OpenStreetMap Nominatim. For a brief moment, I
was beyond excited to discover a hidden tribe of Open Science MOOC-ers
in this small Norwegian town with oh so beautiful fjords and mountains,
but no, this was nothing but a false positive.

\begin{Shaded}
\begin{Highlighting}[]
\CommentTok{# Geocode cleaned locations and save results}
\NormalTok{locations_geo <-}\StringTok{ }\KeywordTok{geocode_OSM}\NormalTok{(data_df}\OperatorTok{$}\NormalTok{location_clean, }\DataTypeTok{as.data.frame =} \OtherTok{TRUE}\NormalTok{, }\DataTypeTok{details =} \OtherTok{TRUE}\NormalTok{)}
\KeywordTok{saveRDS}\NormalTok{(locations_geo, }\StringTok{"locations_geocoded.rds"}\NormalTok{)}
\end{Highlighting}
\end{Shaded}

\begin{Shaded}
\begin{Highlighting}[]
\CommentTok{# Import geocoded data}
\NormalTok{locations_geo <-}\StringTok{ }\KeywordTok{readRDS}\NormalTok{(}\DataTypeTok{file =} \KeywordTok{paste}\NormalTok{(MAIN_DIR,}\StringTok{"locations_geocoded.rds"}\NormalTok{, }\DataTypeTok{sep =} \StringTok{"/"}\NormalTok{))}

\CommentTok{# Clean and merge data}
\NormalTok{locations_df <-}\StringTok{ }\NormalTok{locations_geo }\OperatorTok\StringTok{ }
\StringTok{  }\KeywordTok{rename}\NormalTok{(}\DataTypeTok{location_clean =}\NormalTok{ query) }\OperatorTok\StringTok{ }
\StringTok{  }\KeywordTok{mutate_at}\NormalTok{(}\KeywordTok{vars}\NormalTok{(lat, lon), }\KeywordTok{list}\NormalTok{(}\OperatorTok{~}\StringTok{ }\KeywordTok{ifelse}\NormalTok{(type }\OperatorTok{!=}\StringTok{ "city"}\NormalTok{, }\OtherTok{NA}\NormalTok{, .))) }\OperatorTok
\StringTok{  }\KeywordTok{mutate}\NormalTok{(}\DataTypeTok{display_name =} \KeywordTok{replace}\NormalTok{(display_name, display_name }\OperatorTok{==}\StringTok{ "Norge, Ytterbyvegen, Namsos, Trøndelag, 7810, Norge"}\NormalTok{, }\OtherTok{NA}\NormalTok{)) }\OperatorTok
\StringTok{  }\KeywordTok{select}\NormalTok{(location_clean, display_name, lat, lon)}
  
\NormalTok{data_df }\OperatorTok
\StringTok{  }\KeywordTok{left_join}\NormalTok{(locations_df, }\DataTypeTok{by =} \StringTok{"location_clean"}\NormalTok{) }\OperatorTok\StringTok{ }
\StringTok{  }\KeywordTok{distinct}\NormalTok{()}
\end{Highlighting}
\end{Shaded}

After eliminating our fake Norwegian community members, OpenStreetMap
Nominatim was able to correctly geocode N = 2221 user-provided locations
of the Open Science MOOC followers.

The static map below confirms the results of the previous analysis. Most
followers who provided a geocodable location tend to be clustered in
larger cities in Western Europe and North America. While the map
corroborates the idea of a geographically restricted Open Science
Twitter bubble, it does not for inferences on what causes this pattern.
Potential explanations for this finding could be that users in certain
areas such as China have no legal access to Twitter, prefer other online
communication platforms or have an actual lack of interest in or
knowledge about Open Science.

\begin{Shaded}
\begin{Highlighting}[]
\CommentTok{# Get world map}
\NormalTok{map_world <-}\StringTok{ }\KeywordTok{map_data}\NormalTok{(}\StringTok{"world"}\NormalTok{) }\OperatorTok\StringTok{ }
\StringTok{  }\KeywordTok{filter}\NormalTok{(region }\OperatorTok{!=}\StringTok{ "Antarctica"}\NormalTok{)}

\CommentTok{# Plot followers' locations}
\KeywordTok{ggplot}\NormalTok{() }\OperatorTok{+}
\StringTok{  }\KeywordTok{geom_polygon}\NormalTok{(}\DataTypeTok{data =}\NormalTok{ map_world, }\KeywordTok{aes}\NormalTok{(}\DataTypeTok{x =}\NormalTok{ long, }\DataTypeTok{y =}\NormalTok{ lat, }\DataTypeTok{group =}\NormalTok{ group), }\DataTypeTok{colour =} \StringTok{"gray85"}\NormalTok{, }\DataTypeTok{fill =} \StringTok{"gray80"}\NormalTok{) }\OperatorTok{+}
\StringTok{  }\KeywordTok{geom_point}\NormalTok{(}\DataTypeTok{data =}\NormalTok{ data_df, }\KeywordTok{aes}\NormalTok{(}\DataTypeTok{x =}\NormalTok{ lon, }\DataTypeTok{y =}\NormalTok{ lat), }\DataTypeTok{color =} \StringTok{"#2c3e50"}\NormalTok{, }\DataTypeTok{alpha =} \FloatTok{0.5}\NormalTok{) }\OperatorTok{+}
\StringTok{  }\KeywordTok{ggtitle}\NormalTok{(}\DataTypeTok{label =} \StringTok{"Locations of Open Science Twitter MOOC-ers"}\NormalTok{, }\DataTypeTok{subtitle =} \StringTok{"Profile locations geocoded with OpenStreetMap Nominatim (N = 2221)"}\NormalTok{) }\OperatorTok{+}
\StringTok{  }\KeywordTok{coord_equal}\NormalTok{() }\OperatorTok{+}\StringTok{ }\NormalTok{viz_theme }\OperatorTok{+}\StringTok{ }\KeywordTok{theme}\NormalTok{(}\DataTypeTok{axis.line =} \KeywordTok{element_blank}\NormalTok{(), }
                    \DataTypeTok{axis.text =} \KeywordTok{element_blank}\NormalTok{(), }
                    \DataTypeTok{axis.title =} \KeywordTok{element_blank}\NormalTok{(), }
                    \DataTypeTok{axis.ticks =} \KeywordTok{element_blank}\NormalTok{())}
\end{Highlighting}
\end{Shaded}

\begin{center}\includegraphics{OpenScienceMOOC-follower-analysis_files/figure-latex/locations-static-map1-1} \end{center}

In a slightly more advanced version of this map, I added further
information on the follower count of the geocoded users, which results
in a similar picture: The majority of the influential Open Science
Twitter MOOC-ers are based in Western Europe and the United States. I
will revisit this notion later on in the blog post when conducting a
more fine-grained analysis of the most prominent followers.

\begin{Shaded}
\begin{Highlighting}[]
\CommentTok{# Plot followers' location relative to followers' followers count}
\KeywordTok{ggplot}\NormalTok{() }\OperatorTok{+}
\StringTok{  }\KeywordTok{geom_polygon}\NormalTok{(}\DataTypeTok{data =}\NormalTok{ map_world, }\KeywordTok{aes}\NormalTok{(}\DataTypeTok{x =}\NormalTok{ long, }\DataTypeTok{y =}\NormalTok{ lat, }\DataTypeTok{group =}\NormalTok{ group), }\DataTypeTok{colour =} \StringTok{"gray85"}\NormalTok{, }\DataTypeTok{fill =} \StringTok{"gray80"}\NormalTok{) }\OperatorTok{+}
\StringTok{  }\KeywordTok{geom_point}\NormalTok{(}\DataTypeTok{data =}\NormalTok{ data_df, }\KeywordTok{aes}\NormalTok{(}\DataTypeTok{x =}\NormalTok{ lon, }\DataTypeTok{y =}\NormalTok{ lat, }\DataTypeTok{size =}\NormalTok{ followersCount), }\DataTypeTok{color =} \StringTok{"#2c3e50"}\NormalTok{, }\DataTypeTok{alpha =} \FloatTok{0.5}\NormalTok{) }\OperatorTok{+}
\StringTok{  }\KeywordTok{ggtitle}\NormalTok{(}\DataTypeTok{label =} \StringTok{"Locations of Open Science Twitter MOOC-ers"}\NormalTok{, }\DataTypeTok{subtitle =} \StringTok{"Marker sizes relative to followers' followers counts (N = 2220)"}\NormalTok{) }\OperatorTok{+}
\StringTok{  }\KeywordTok{scale_size_continuous}\NormalTok{(}\DataTypeTok{range =} \KeywordTok{c}\NormalTok{(}\DecValTok{1}\NormalTok{, }\DecValTok{6}\NormalTok{), }\DataTypeTok{limits =} \KeywordTok{c}\NormalTok{(}\DecValTok{0}\NormalTok{, }\DecValTok{300000}\NormalTok{),}
                        \DataTypeTok{breaks =} \KeywordTok{c}\NormalTok{(}\DecValTok{0}\NormalTok{, }\DecValTok{1000}\NormalTok{, }\DecValTok{10000}\NormalTok{, }\DecValTok{100000}\NormalTok{, }\DecValTok{200000}\NormalTok{, }\DecValTok{300000}\NormalTok{), }
                        \DataTypeTok{labels =} \ControlFlowTok{function}\NormalTok{(x) }\KeywordTok{format}\NormalTok{(x, }\DataTypeTok{scientific =} \OtherTok{FALSE}\NormalTok{)) }\OperatorTok{+}
\StringTok{  }\KeywordTok{labs}\NormalTok{(}\DataTypeTok{size =} \StringTok{"Number of followers' followers"}\NormalTok{) }\OperatorTok{+}\StringTok{ }\KeywordTok{coord_equal}\NormalTok{() }\OperatorTok{+}
\StringTok{  }\NormalTok{viz_theme }\OperatorTok{+}\StringTok{ }\KeywordTok{theme}\NormalTok{(}\DataTypeTok{legend.position =} \StringTok{"none"}\NormalTok{, }
                    \DataTypeTok{axis.line =} \KeywordTok{element_blank}\NormalTok{(), }
                    \DataTypeTok{axis.text =} \KeywordTok{element_blank}\NormalTok{(), }
                    \DataTypeTok{axis.title =} \KeywordTok{element_blank}\NormalTok{(), }
                    \DataTypeTok{axis.ticks =} \KeywordTok{element_blank}\NormalTok{())}
\end{Highlighting}
\end{Shaded}

\begin{center}\includegraphics{OpenScienceMOOC-follower-analysis_files/figure-latex/locations-static-map2-1} \end{center}

The static maps above can be supplemented by creating an interactive map
using \texttt{leaflet}, which allows to zoom into all areas and take a
closer look at specific countries and regions.

\begin{Shaded}
\begin{Highlighting}[]
\CommentTok{# Plot followers' locations using leaflet}
\KeywordTok{leaflet}\NormalTok{(data_df, }\DataTypeTok{width =} \StringTok{"100%"}\NormalTok{) }\OperatorTok\StringTok{ }
\StringTok{  }\KeywordTok{addProviderTiles}\NormalTok{(}\StringTok{"CartoDB.DarkMatter"}\NormalTok{) }\OperatorTok\StringTok{ }
\StringTok{  }\KeywordTok{addCircleMarkers}\NormalTok{(}\OperatorTok{~}\NormalTok{lon,}
                   \OperatorTok{~}\NormalTok{lat,}
                   \DataTypeTok{color =} \StringTok{"#9eb4ca"}\NormalTok{,}
                   \DataTypeTok{radius =} \FloatTok{1.5}\NormalTok{)}
\end{Highlighting}
\end{Shaded}

\hypertarget{htmlwidget-3a90feb9eef4ce658233}{}

\subsection{Diversity and inclusion}\label{diversity-and-inclusion}

After tracking down their locations, I was interested in the gender
diversity of the Open Science MOOC followers since one of its objectives
is to be an inclusive and welcoming platform for people of all colors
and genders.

``How does this work with Twitter data where no information on users'
genders is available?'' you might ask. A while ago I worked on a project
using bibliometric data to analyze gender differences in computational
social science publications (that in the end sadly never saw the light
of the day) and stumbled upon
\href{https://genderize.io/}{genderize.io}. Their API draws on a
database containing over 200,000 distinct names from 79 countries and 89
languages which were obtained from user profiles across various social
networks to determine people's gender based on their first name. While
this approach only allows for qualified guesses rather than perfectly
robust results, it still provides a rough estimate of the gender
distribution among the Open Science MOOC-ers.

Before determining each followers gender, I preprocessed all names by
again removing hashtags, mentions, punctuation, numbers, and non-ASCII
characters - yes, people on Twitter are amazingly creative when it comes
to their online identities - as well as academic titles such as Dr, PhD
or Prof.

\begin{Shaded}
\begin{Highlighting}[]
\CommentTok{# Clean name and split name into first and last name}
\NormalTok{## Note: Names consisting of non-ASCII characters are removed.}
\NormalTok{data_df }\OperatorTok
\StringTok{  }\KeywordTok{mutate}\NormalTok{(}\DataTypeTok{name_clean =} \KeywordTok{gsub}\NormalTok{(}\StringTok{"#}\CharTok{\textbackslash{}\textbackslash{}}\StringTok{S+"}\NormalTok{, }\StringTok{""}\NormalTok{, name)) }\OperatorTok\StringTok{ }
\StringTok{  }\KeywordTok{mutate}\NormalTok{(}\DataTypeTok{name_clean =} \KeywordTok{gsub}\NormalTok{(}\StringTok{"@}\CharTok{\textbackslash{}\textbackslash{}}\StringTok{S+"}\NormalTok{, }\StringTok{""}\NormalTok{, name_clean)) }\OperatorTok\StringTok{ }
\StringTok{  }\KeywordTok{mutate}\NormalTok{(}\DataTypeTok{name_clean =} \KeywordTok{gsub}\NormalTok{(}\StringTok{"[^}\CharTok{\textbackslash{}x01}\StringTok{-}\CharTok{\textbackslash{}x7F}\StringTok{]"}\NormalTok{, }\StringTok{""}\NormalTok{, name_clean)) }\OperatorTok\StringTok{ }\CommentTok{# remove non-ASCII characters}
\StringTok{  }\KeywordTok{mutate}\NormalTok{(}\DataTypeTok{name_clean =} \KeywordTok{gsub}\NormalTok{(}\StringTok{"[^[:alnum:][:space:]}\CharTok{\textbackslash{}\textbackslash{}}\StringTok{-]"}\NormalTok{, }\StringTok{""}\NormalTok{, name_clean)) }\OperatorTok\StringTok{ }
\StringTok{  }\KeywordTok{mutate}\NormalTok{(}\DataTypeTok{name_clean =} \KeywordTok{gsub}\NormalTok{(}\StringTok{"^[[:punct:]]"}\NormalTok{, }\StringTok{""}\NormalTok{, name_clean)) }\OperatorTok\StringTok{ }
\StringTok{  }\KeywordTok{mutate}\NormalTok{(}\DataTypeTok{name_clean =} \KeywordTok{trimws}\NormalTok{(}\KeywordTok{gsub}\NormalTok{(}\StringTok{"}\CharTok{\textbackslash{}\textbackslash{}}\StringTok{w*[0-9]+}\CharTok{\textbackslash{}\textbackslash{}}\StringTok{w*}\CharTok{\textbackslash{}\textbackslash{}}\StringTok{s*"}\NormalTok{, }\StringTok{""}\NormalTok{, name_clean))) }\OperatorTok
\StringTok{  }\KeywordTok{mutate}\NormalTok{(}\DataTypeTok{name_clean =} \KeywordTok{gsub}\NormalTok{(}\StringTok{" *}\CharTok{\textbackslash{}\textbackslash{}}\StringTok{b[[:alpha:]]\{1\}}\CharTok{\textbackslash{}\textbackslash{}}\StringTok{b *"}\NormalTok{, }\StringTok{" "}\NormalTok{, name_clean)) }\OperatorTok\StringTok{ }
\StringTok{  }\KeywordTok{mutate}\NormalTok{(}\DataTypeTok{name_clean =} \KeywordTok{gsub}\NormalTok{(}\StringTok{"Dr}\CharTok{\textbackslash{}\textbackslash{}}\StringTok{s*"}\NormalTok{, }\StringTok{""}\NormalTok{, name_clean), }\DataTypeTok{name_clean =} \KeywordTok{gsub}\NormalTok{(}\StringTok{"PhD}\CharTok{\textbackslash{}\textbackslash{}}\StringTok{s*"}\NormalTok{, }\StringTok{""}\NormalTok{, name_clean), }
         \DataTypeTok{name_clean =} \KeywordTok{gsub}\NormalTok{(}\StringTok{"Prof}\CharTok{\textbackslash{}\textbackslash{}}\StringTok{s*"}\NormalTok{, }\StringTok{""}\NormalTok{, name_clean), }\DataTypeTok{name_clean =} \KeywordTok{gsub}\NormalTok{(}\StringTok{"MSc}\CharTok{\textbackslash{}\textbackslash{}}\StringTok{s*"}\NormalTok{, }\StringTok{""}\NormalTok{, name_clean), }
         \DataTypeTok{name_clean =} \KeywordTok{gsub}\NormalTok{(}\StringTok{"MA}\CharTok{\textbackslash{}\textbackslash{}}\StringTok{s*"}\NormalTok{, }\StringTok{""}\NormalTok{, name_clean), }\DataTypeTok{name_clean =} \KeywordTok{gsub}\NormalTok{(}\StringTok{"MD}\CharTok{\textbackslash{}\textbackslash{}}\StringTok{s*"}\NormalTok{, }\StringTok{""}\NormalTok{, name_clean)) }\OperatorTok
\StringTok{  }\KeywordTok{mutate}\NormalTok{(}\DataTypeTok{name_clean =} \KeywordTok{str_trim}\NormalTok{(name_clean, }\DataTypeTok{side =} \StringTok{"both"}\NormalTok{)) }\OperatorTok
\StringTok{  }\KeywordTok{mutate}\NormalTok{(}\DataTypeTok{name_clean =} \KeywordTok{gsub}\NormalTok{(}\StringTok{"^$"}\NormalTok{, }\OtherTok{NA}\NormalTok{, }\KeywordTok{trimws}\NormalTok{(name_clean))) }\OperatorTok
\StringTok{  }\KeywordTok{separate}\NormalTok{(name_clean, }\KeywordTok{c}\NormalTok{(}\StringTok{"first_name"}\NormalTok{, }\StringTok{"last_name"}\NormalTok{), }\DataTypeTok{sep =} \StringTok{" "}\NormalTok{, }\DataTypeTok{remove =} \OtherTok{FALSE}\NormalTok{)}
\end{Highlighting}
\end{Shaded}

Prior to predicting the Open Science MOOC followers' genders with
\texttt{genderizeR} - a package accesses the genderize.io API from
within R - I drew a random sample of N = 1000 followers with non-missing
first names as the free API is limited to classifying 1000 names per
day.

\begin{Shaded}
\begin{Highlighting}[]
\CommentTok{# Draw random sample of followers}
\KeywordTok{set.seed}\NormalTok{(}\DecValTok{42}\NormalTok{)}
\NormalTok{df_sample <-}\StringTok{ }\NormalTok{data_df }\OperatorTok
\StringTok{  }\KeywordTok{filter}\NormalTok{(}\OperatorTok{!}\KeywordTok{is.na}\NormalTok{(first_name)) }\OperatorTok
\StringTok{  }\KeywordTok{sample_n}\NormalTok{(}\DecValTok{1000}\NormalTok{)}

\CommentTok{# Predict and genderize names}
\NormalTok{givenNames <-}\StringTok{ }\KeywordTok{findGivenNames}\NormalTok{(df_sample}\OperatorTok{$}\NormalTok{first_name, }\DataTypeTok{progress =} \OtherTok{FALSE}\NormalTok{)}
\NormalTok{followers_gender <-}\StringTok{ }\KeywordTok{genderize}\NormalTok{(df_sample}\OperatorTok{$}\NormalTok{first_name, }\DataTypeTok{genderDB =}\NormalTok{ givenNames, }\DataTypeTok{progress =} \OtherTok{FALSE}\NormalTok{)}

\CommentTok{# Export data}
\KeywordTok{saveRDS}\NormalTok{(followers_gender, }\StringTok{"followers_gender.rds"}\NormalTok{)}
\end{Highlighting}
\end{Shaded}

The following plot shows the gender demographics of the sampled
followers. The genderize.io API confidently classified 403 first names
as male and 381 as female. 216 names could not be classified and thus
resulted in missing values. Hence, there tend to be slightly more male
than female followers, but overall it is pretty balanced.

\begin{Shaded}
\begin{Highlighting}[]
\CommentTok{# Import genderized data}
\NormalTok{followers_gender <-}\StringTok{ }\KeywordTok{readRDS}\NormalTok{(}\DataTypeTok{file =} \KeywordTok{paste}\NormalTok{(MAIN_DIR,}\StringTok{"followers_gender.rds"}\NormalTok{, }\DataTypeTok{sep =} \StringTok{"/"}\NormalTok{))}

\CommentTok{# Merge with main data frame and change gender for "NA" to NA}
\NormalTok{data_df }\OperatorTok\StringTok{ }
\StringTok{  }\KeywordTok{left_join}\NormalTok{(followers_gender, }\DataTypeTok{by =} \KeywordTok{c}\NormalTok{(}\StringTok{"first_name"}\NormalTok{ =}\StringTok{ "text"}\NormalTok{)) }\OperatorTok
\StringTok{  }\KeywordTok{select}\NormalTok{(}\OperatorTok{-}\NormalTok{genderIndicators) }\OperatorTok
\StringTok{  }\KeywordTok{mutate}\NormalTok{(}\DataTypeTok{gender =} \KeywordTok{replace}\NormalTok{(gender, gender }\OperatorTok{==}\StringTok{ "NA"}\NormalTok{, }\OtherTok{NA}\NormalTok{)) }\OperatorTok
\StringTok{  }\KeywordTok{distinct}\NormalTok{()}

\CommentTok{# Plot gender distribution using genderized data}
\NormalTok{followers_gender }\OperatorTok\StringTok{ }
\StringTok{  }\KeywordTok{mutate}\NormalTok{(}\DataTypeTok{gender =} \KeywordTok{replace}\NormalTok{(gender, gender }\OperatorTok{==}\StringTok{ "NA"}\NormalTok{, }\OtherTok{NA}\NormalTok{)) }\OperatorTok
\StringTok{  }\KeywordTok{filter}\NormalTok{(}\OperatorTok{!}\KeywordTok{is.na}\NormalTok{(gender)) }\OperatorTok
\StringTok{  }\KeywordTok{ggplot}\NormalTok{(}\DataTypeTok{mapping =} \KeywordTok{aes}\NormalTok{(}\DataTypeTok{x =}\NormalTok{ gender, }\DataTypeTok{fill =}\NormalTok{ gender)) }\OperatorTok{+}
\StringTok{  }\KeywordTok{scale_fill_manual}\NormalTok{(}\StringTok{" "}\NormalTok{, }\DataTypeTok{values =} \KeywordTok{c}\NormalTok{(}\StringTok{"#6c7a89"}\NormalTok{, }\StringTok{"#22313f"}\NormalTok{)) }\OperatorTok{+}
\StringTok{  }\KeywordTok{scale_x_discrete}\NormalTok{(}\DataTypeTok{labels =} \KeywordTok{c}\NormalTok{(}\StringTok{"Female"}\NormalTok{, }\StringTok{"Male"}\NormalTok{)) }\OperatorTok{+}
\StringTok{  }\NormalTok{viz_theme }\OperatorTok{+}\StringTok{ }\KeywordTok{ylab}\NormalTok{(}\StringTok{""}\NormalTok{) }\OperatorTok{+}\StringTok{ }\KeywordTok{xlab}\NormalTok{(}\StringTok{""}\NormalTok{) }\OperatorTok{+}
\StringTok{  }\KeywordTok{geom_bar}\NormalTok{(}\DataTypeTok{alpha =} \FloatTok{0.8}\NormalTok{, }\DataTypeTok{width =} \FloatTok{0.5}\NormalTok{) }\OperatorTok{+}
\StringTok{  }\KeywordTok{ggtitle}\NormalTok{(}\DataTypeTok{label =} \StringTok{"Count of Open Science Twitter MOOC-ers by gender"}\NormalTok{, }\DataTypeTok{subtitle =} \StringTok{"Gender predicted with genderize.io (random sample of N = 1000 followers)"}\NormalTok{) }\OperatorTok{+}
\StringTok{  }\KeywordTok{theme}\NormalTok{(}\DataTypeTok{legend.position =} \StringTok{"none"}\NormalTok{)}
\end{Highlighting}
\end{Shaded}

\begin{center}\includegraphics{OpenScienceMOOC-follower-analysis_files/figure-latex/gender-1} \end{center}

To get a better picture of how inclusive the Open Science MOOC Twitter
community is, I additionally classified all followers based on their
respective account status. This approach was adapted from a blog post on
\href{https://correlaid.org/blog/we2-twitter-analysis/}{analyzing social
movements on Twitter} that I co-wrote with my fellow CorrelAider
\href{http://konstantin.gavras.de/}{Konstantin Gavras} for the European
elections in May 2019.

All followers of the Open Science MOOC were classified as follows: 1.
Verified account: Account is officially verified by Twitter (i.e.~of
public interest) 2. Influencer: Account has at least 500 followers and
at least thrice as many followers than friends 3- Verified influencer:
Account is both officially verified and classified as an influencer 4.
Personal account: Account is neither verified nor classified as an
influencer

As can be seen in the plot below, most accounts are unverified personal
ones, making the Open Science MOOC Twitter community a very inclusive
digital space.

\begin{Shaded}
\begin{Highlighting}[]
\CommentTok{# Add influencer status}
\NormalTok{## Approach adapted from https://correlaid.org/blog/we2-twitter-analysis/.}
\NormalTok{data_df}\OperatorTok{$}\NormalTok{influencer <-}\StringTok{ }\KeywordTok{ifelse}\NormalTok{(data_df}\OperatorTok{$}\NormalTok{followersCount }\OperatorTok{>=}\StringTok{ }\DecValTok{500} \OperatorTok{&}\StringTok{ }\NormalTok{data_df}\OperatorTok{$}\NormalTok{followersCount }\OperatorTok{>=}\StringTok{ }\DecValTok{3} \OperatorTok{*}\StringTok{ }\NormalTok{data_df}\OperatorTok{$}\NormalTok{friendsCount, }\StringTok{'YES'}\NormalTok{, }\StringTok{'NO'}\NormalTok{)}

\CommentTok{# Classify accounts into differenct user categories based on verification status and influencer status}
\NormalTok{data_df}\OperatorTok{$}\NormalTok{category <-}\StringTok{ }\KeywordTok{ifelse}\NormalTok{(data_df}\OperatorTok{$}\NormalTok{verified }\OperatorTok{==}\StringTok{ }\OtherTok{FALSE} \OperatorTok{&}\StringTok{ }\NormalTok{data_df}\OperatorTok{$}\NormalTok{influencer }\OperatorTok{==}\StringTok{ 'YES'}\NormalTok{, }\StringTok{'influencer'}\NormalTok{,}
                           \KeywordTok{ifelse}\NormalTok{(data_df}\OperatorTok{$}\NormalTok{verified }\OperatorTok{==}\StringTok{ }\OtherTok{TRUE} \OperatorTok{&}\StringTok{ }\NormalTok{data_df}\OperatorTok{$}\NormalTok{influencer }\OperatorTok{==}\StringTok{ 'NO'}\NormalTok{, }\StringTok{'verified'}\NormalTok{,}
                                  \KeywordTok{ifelse}\NormalTok{(data_df}\OperatorTok{$}\NormalTok{verified }\OperatorTok{==}\StringTok{ }\OtherTok{TRUE} \OperatorTok{&}\StringTok{ }\NormalTok{data_df}\OperatorTok{$}\NormalTok{influencer }\OperatorTok{==}\StringTok{ 'YES'}\NormalTok{, }\StringTok{'verified_influencer'}\NormalTok{, }\StringTok{'personal'}\NormalTok{)))}

\CommentTok{# Plot number of accounts by status category}
\NormalTok{## Credits: Colors taken from https://www.colorcombos.com/color-schemes/554/ColorCombo554.html.}
\NormalTok{data_df }\OperatorTok\StringTok{ }
\StringTok{  }\KeywordTok{select}\NormalTok{(screenName, category) }\OperatorTok
\StringTok{  }\KeywordTok{unique}\NormalTok{() }\OperatorTok
\StringTok{  }\KeywordTok{ggplot}\NormalTok{(}\DataTypeTok{mapping =} \KeywordTok{aes}\NormalTok{(}\DataTypeTok{x =}\NormalTok{ category, }\DataTypeTok{fill =}\NormalTok{ category)) }\OperatorTok{+}
\StringTok{  }\KeywordTok{scale_fill_manual}\NormalTok{(}\StringTok{" "}\NormalTok{, }\DataTypeTok{values =} \KeywordTok{c}\NormalTok{(}\StringTok{"#C0392B"}\NormalTok{, }\StringTok{"#2C3E50"}\NormalTok{, }\StringTok{"#16A085"}\NormalTok{, }\StringTok{"#F1C40F"}\NormalTok{)) }\OperatorTok{+}
\StringTok{  }\KeywordTok{scale_x_discrete}\NormalTok{(}\DataTypeTok{labels =} \KeywordTok{c}\NormalTok{(}\StringTok{"Influencer"}\NormalTok{, }\StringTok{"Personal account"}\NormalTok{, }\StringTok{"Verified account"}\NormalTok{, }\StringTok{"Verified influencer"}\NormalTok{)) }\OperatorTok{+}
\StringTok{  }\NormalTok{viz_theme }\OperatorTok{+}\StringTok{ }\KeywordTok{ylab}\NormalTok{(}\StringTok{""}\NormalTok{) }\OperatorTok{+}\StringTok{ }\KeywordTok{xlab}\NormalTok{(}\StringTok{""}\NormalTok{) }\OperatorTok{+}
\StringTok{  }\KeywordTok{geom_bar}\NormalTok{(}\DataTypeTok{alpha =} \FloatTok{0.8}\NormalTok{, }\DataTypeTok{width =} \FloatTok{0.5}\NormalTok{) }\OperatorTok{+}
\StringTok{  }\KeywordTok{ggtitle}\NormalTok{(}\DataTypeTok{label =} \StringTok{'Number of Open Science Twitter MOOC-ers by account status'}\NormalTok{, }\DataTypeTok{subtitle =} \StringTok{"Influencer: at least 500 followers and at least thrice as many followers than friends }\CharTok{\textbackslash{}n}\StringTok{Verified: official verification status"}\NormalTok{) }\OperatorTok{+}\StringTok{ }\KeywordTok{ylim}\NormalTok{(}\DecValTok{0}\NormalTok{, }\DecValTok{6000}\NormalTok{) }\OperatorTok{+}
\StringTok{  }\KeywordTok{theme}\NormalTok{(}\DataTypeTok{legend.position =} \StringTok{"none"}\NormalTok{)}
\end{Highlighting}
\end{Shaded}

\begin{center}\includegraphics{OpenScienceMOOC-follower-analysis_files/figure-latex/account-status-1} \end{center}

\subsection{Followers or leaders?}\label{followers-or-leaders}

Now that we have seen that there are only few verified accounts and
influencers being part of the community, who are the followers with the
largest follower base? Here you have an overview of the most influential
Open Science MOOC-ers:

\begin{Shaded}
\begin{Highlighting}[]
\CommentTok{# Plot most influential followers}
\NormalTok{data_df }\OperatorTok
\StringTok{  }\KeywordTok{top_n}\NormalTok{(}\DecValTok{20}\NormalTok{, followersCount) }\OperatorTok
\StringTok{  }\KeywordTok{mutate}\NormalTok{(}\DataTypeTok{screenName =} \KeywordTok{reorder}\NormalTok{(screenName, followersCount)) }\OperatorTok
\StringTok{  }\KeywordTok{ggplot}\NormalTok{(}\KeywordTok{aes}\NormalTok{(screenName, followersCount, }\DataTypeTok{label =}\NormalTok{ followersCount)) }\OperatorTok{+}
\StringTok{  }\KeywordTok{geom_bar}\NormalTok{(}\DataTypeTok{stat =} \StringTok{"identity"}\NormalTok{, }\DataTypeTok{width =} \FloatTok{0.5}\NormalTok{, }\DataTypeTok{alpha =} \FloatTok{0.8}\NormalTok{, }\DataTypeTok{color =} \StringTok{"#2c3e50"}\NormalTok{, }\DataTypeTok{fill =} \StringTok{"#2c3e50"}\NormalTok{) }\OperatorTok{+}
\StringTok{  }\KeywordTok{xlab}\NormalTok{(}\StringTok{""}\NormalTok{) }\OperatorTok{+}\StringTok{ }\KeywordTok{ylab}\NormalTok{(}\StringTok{""}\NormalTok{) }\OperatorTok{+}\StringTok{ }\KeywordTok{ggtitle}\NormalTok{(}\StringTok{"Most influential Open Science MOOC-ers"}\NormalTok{, }\DataTypeTok{subtitle =} \StringTok{"Followers count"}\NormalTok{) }\OperatorTok{+}
\StringTok{  }\KeywordTok{scale_y_continuous}\NormalTok{(}\DataTypeTok{labels =} \ControlFlowTok{function}\NormalTok{(x) }\KeywordTok{format}\NormalTok{(x, }\DataTypeTok{scientific =} \OtherTok{FALSE}\NormalTok{)) }\OperatorTok{+}
\StringTok{  }\KeywordTok{coord_flip}\NormalTok{() }\OperatorTok{+}\StringTok{ }\NormalTok{viz_theme }
\end{Highlighting}
\end{Shaded}

\begin{center}\includegraphics{OpenScienceMOOC-follower-analysis_files/figure-latex/influential-followers-1} \end{center}

To get a more detailed picture, let us also have a look at the most
active followers. Here, I followed Shirin's example and normalized the
number of statuses, meaning tweets and retweets, each account shared by
the number of days the account existed.

\begin{Shaded}
\begin{Highlighting}[]
\CommentTok{# Calculate each follower's average number of tweets per day}
\NormalTok{## Approach adapted from https://shiring.github.io/text_analysis/2017/06/28/twitter_post.}
\NormalTok{data_df }\OperatorTok
\StringTok{  }\KeywordTok{mutate}\NormalTok{(}\DataTypeTok{created_date =} \KeywordTok{as.Date}\NormalTok{(created, }\DataTypeTok{format =} \StringTok{"%Y-%m-%d"}\NormalTok{),}
         \DataTypeTok{today =} \KeywordTok{as.Date}\NormalTok{(}\StringTok{"2019-05-28"}\NormalTok{, }\DataTypeTok{format =} \StringTok{"%Y-%m-%d"}\NormalTok{),}
         \DataTypeTok{days =} \KeywordTok{as.numeric}\NormalTok{(today }\OperatorTok{-}\StringTok{ }\NormalTok{created_date),}
         \DataTypeTok{statuses_day =}\NormalTok{ statusesCount }\OperatorTok{/}\StringTok{ }\NormalTok{days) }\OperatorTok
\StringTok{  }\KeywordTok{select}\NormalTok{(}\OperatorTok{-}\NormalTok{today)}

\CommentTok{# Plot most active followers}
\NormalTok{data_df }\OperatorTok
\StringTok{  }\KeywordTok{top_n}\NormalTok{(}\DecValTok{20}\NormalTok{, statuses_day) }\OperatorTok
\StringTok{  }\KeywordTok{mutate}\NormalTok{(}\DataTypeTok{screenName =} \KeywordTok{reorder}\NormalTok{(screenName, statuses_day)) }\OperatorTok
\StringTok{  }\KeywordTok{ggplot}\NormalTok{(}\KeywordTok{aes}\NormalTok{(screenName, statuses_day, }\DataTypeTok{label =}\NormalTok{ statuses_day)) }\OperatorTok{+}
\StringTok{  }\KeywordTok{geom_bar}\NormalTok{(}\DataTypeTok{stat =} \StringTok{"identity"}\NormalTok{, }\DataTypeTok{width =} \FloatTok{0.5}\NormalTok{, }\DataTypeTok{alpha =} \FloatTok{0.8}\NormalTok{, }\DataTypeTok{color =} \StringTok{"#2c3e50"}\NormalTok{, }\DataTypeTok{fill =} \StringTok{"#2c3e50"}\NormalTok{) }\OperatorTok{+}
\StringTok{  }\KeywordTok{xlab}\NormalTok{(}\StringTok{""}\NormalTok{) }\OperatorTok{+}\StringTok{ }\KeywordTok{ylab}\NormalTok{(}\StringTok{""}\NormalTok{) }\OperatorTok{+}\StringTok{ }\KeywordTok{ggtitle}\NormalTok{(}\StringTok{"Most active Open Science MOOC-ers"}\NormalTok{, }\DataTypeTok{subtitle =} \StringTok{"Average number of tweets per day"}\NormalTok{) }\OperatorTok{+}
\StringTok{  }\KeywordTok{scale_y_continuous}\NormalTok{(}\DataTypeTok{labels =} \ControlFlowTok{function}\NormalTok{(x) }\KeywordTok{format}\NormalTok{(x, }\DataTypeTok{scientific =} \OtherTok{FALSE}\NormalTok{)) }\OperatorTok{+}\StringTok{ }\KeywordTok{ylim}\NormalTok{(}\DecValTok{0}\NormalTok{, }\DecValTok{200}\NormalTok{) }\OperatorTok{+}
\StringTok{  }\KeywordTok{coord_flip}\NormalTok{() }\OperatorTok{+}\StringTok{ }\NormalTok{viz_theme }
\end{Highlighting}
\end{Shaded}

\begin{center}\includegraphics{OpenScienceMOOC-follower-analysis_files/figure-latex/active-followers-1} \end{center}

And yes, there is a correlation between the number of followers and the
number of tweets, indicating that Open Science MOOC-ers with many
followers also tend to tweet more often.

\begin{Shaded}
\begin{Highlighting}[]
\NormalTok{data_df }\OperatorTok
\StringTok{  }\KeywordTok{ggplot}\NormalTok{(}\KeywordTok{aes}\NormalTok{(}\DataTypeTok{x =} \KeywordTok{log2}\NormalTok{(followersCount), }\DataTypeTok{y =} \KeywordTok{log2}\NormalTok{(statuses_day), }\DataTypeTok{color =}\NormalTok{ days)) }\OperatorTok{+}
\StringTok{    }\KeywordTok{geom_smooth}\NormalTok{(}\DataTypeTok{method =} \StringTok{"lm"}\NormalTok{, }\DataTypeTok{color =} \StringTok{"grey50"}\NormalTok{, }\DataTypeTok{fill =} \StringTok{"grey90"}\NormalTok{, }\DataTypeTok{alpha =} \FloatTok{0.8}\NormalTok{) }\OperatorTok{+}
\StringTok{    }\KeywordTok{geom_point}\NormalTok{(}\DataTypeTok{alpha =} \FloatTok{0.8}\NormalTok{) }\OperatorTok{+}
\StringTok{  }\KeywordTok{scale_color_continuous}\NormalTok{(}\StringTok{"Number of days since }\CharTok{\textbackslash{}n}\StringTok{account was created"}\NormalTok{, }\DataTypeTok{low =} \StringTok{"#91aac3"}\NormalTok{, }\DataTypeTok{high =} \StringTok{"#2C3E50"}\NormalTok{) }\OperatorTok{+}
\StringTok{    }\KeywordTok{xlab}\NormalTok{(}\StringTok{"Number of followers (log2)"}\NormalTok{) }\OperatorTok{+}\StringTok{ }\KeywordTok{ylab}\NormalTok{(}\StringTok{"Average number of tweets per day (log2)"}\NormalTok{) }\OperatorTok{+}\StringTok{ }\KeywordTok{ggtitle}\NormalTok{(}\StringTok{"Correlation between the Open Science MOOC-ers followers count and tweets per day"}\NormalTok{, }\DataTypeTok{subtitle =} \StringTok{" "}\NormalTok{) }\OperatorTok{+}
\StringTok{    }\NormalTok{viz_theme}
\end{Highlighting}
\end{Shaded}

\begin{center}\includegraphics{OpenScienceMOOC-follower-analysis_files/figure-latex/correlation-1} \end{center}

\subsection{Activities, interests, and
opinions}\label{activities-interests-and-opinions}

After these insights into the locations and demographics of the Open
Science Twitter MOOC-ers, we now analyze the contents of their own
profile descriptions.

You guessed correctly, these unstrctured texts need some preprocessing
as well. In this case, preprocessing translates to removing URLS,
punctuation, non-alphabetic characters, leading and trailing whitespace
as well as English, German, French, and Spanish stopwords. @-mentions
and hashtags are intentionally kept since they are parts of the actual
analysis. Profile descriptions originally consist of a sequence of
strings and after cleaning them, I split them into single words. This
process is called \emph{tokenization} in natural language processing.

I also stemmed these preprocessed words using
\href{http://www.cs.toronto.edu/~frank/csc2501/Readings/R2_Porter/Porter-1980.pdf}{Martin
Porter's stemming algorithm} for collapsing words to a common root. This
was done with the \texttt{SnowballC} package to help facilitate
comparison between followers' vocabulary. Alternatively, I also tried
lemmatization with \texttt{textstem}, that is removing inflectional
endings and grouping words into a single base form (the so-called
\emph{lemma}), but I was not too happy with the results. So in the end I
just went with stemming.

\begin{Shaded}
\begin{Highlighting}[]
\CommentTok{# Get German, French, and Spanish stop words}
\NormalTok{stop_german <-}\StringTok{ }\KeywordTok{data.frame}\NormalTok{(}\DataTypeTok{word =}\NormalTok{ stopwords}\OperatorTok{::}\KeywordTok{stopwords}\NormalTok{(}\StringTok{"de"}\NormalTok{), }\DataTypeTok{stringsAsFactors =}\NormalTok{ F)}
\NormalTok{stop_french <-}\StringTok{ }\KeywordTok{data.frame}\NormalTok{(}\DataTypeTok{word =}\NormalTok{ stopwords}\OperatorTok{::}\KeywordTok{stopwords}\NormalTok{(}\StringTok{"fr"}\NormalTok{), }\DataTypeTok{stringsAsFactors =}\NormalTok{ F)}
\NormalTok{stop_espanol <-}\StringTok{ }\KeywordTok{data.frame}\NormalTok{(}\DataTypeTok{word =}\NormalTok{ stopwords}\OperatorTok{::}\KeywordTok{stopwords}\NormalTok{(}\StringTok{"es"}\NormalTok{), }\DataTypeTok{stringsAsFactors =}\NormalTok{ F)}

\CommentTok{# Specify pattern for tokenization}
\NormalTok{## Regex adapted from https://pushpullfork.com/mining-twitter-data-tidy-text-tags/.}
\NormalTok{pattern_words <-}\StringTok{ "([^A-Za-z_}\CharTok{\textbackslash{}\textbackslash{}}\StringTok{d#@]|'(?![A-Za-z_}\CharTok{\textbackslash{}\textbackslash{}}\StringTok{d#@]))"}

\CommentTok{# Tidy descriptions}
\NormalTok{desc_tidy <-}\StringTok{ }\NormalTok{data_df }\OperatorTok
\StringTok{  }\KeywordTok{mutate}\NormalTok{(}\DataTypeTok{description_clean =} \KeywordTok{gsub}\NormalTok{(}\StringTok{"^$"}\NormalTok{, }\OtherTok{NA}\NormalTok{, }\KeywordTok{trimws}\NormalTok{(description))) }\OperatorTok
\StringTok{  }\KeywordTok{mutate}\NormalTok{(}\DataTypeTok{description_clean =} \KeywordTok{gsub}\NormalTok{(}\StringTok{"http}\CharTok{\textbackslash{}\textbackslash{}}\StringTok{S+}\CharTok{\textbackslash{}\textbackslash{}}\StringTok{s*"}\NormalTok{, }\StringTok{""}\NormalTok{, description_clean)) }\OperatorTok\StringTok{ }\CommentTok{# remove URLS }
\StringTok{  }\KeywordTok{unnest_tokens}\NormalTok{(word, description_clean, }\DataTypeTok{token =} \StringTok{"regex"}\NormalTok{, }\DataTypeTok{pattern =}\NormalTok{ pattern_words) }\OperatorTok
\StringTok{  }\KeywordTok{filter}\NormalTok{(}\OperatorTok{!}\NormalTok{word }\OperatorTok\StringTok{ }\NormalTok{stop_words}\OperatorTok{$}\NormalTok{word, }\OperatorTok{!}\NormalTok{word }\OperatorTok\StringTok{ }\NormalTok{stop_german}\OperatorTok{$}\NormalTok{word, }\OperatorTok{!}\NormalTok{word }\OperatorTok\StringTok{ }\NormalTok{stop_french}\OperatorTok{$}\NormalTok{word, }
         \OperatorTok{!}\NormalTok{word }\OperatorTok\StringTok{ }\NormalTok{stop_espanol}\OperatorTok{$}\NormalTok{word, }\KeywordTok{str_detect}\NormalTok{(word, }\StringTok{"[a-z]"}\NormalTok{)) }\OperatorTok\StringTok{ }\CommentTok{# remove stop words}
\StringTok{  }\KeywordTok{mutate}\NormalTok{(}\DataTypeTok{word_stem =} \KeywordTok{wordStem}\NormalTok{(word)) }\OperatorTok\StringTok{ }\CommentTok{# stem words}
\StringTok{  }\KeywordTok{select}\NormalTok{(word, word_stem, screenName)}
\end{Highlighting}
\end{Shaded}

In a first step, I filtered the most common words the Open Science MOOC
followers used in their profile descriptions. The plot below shows that
words referring to all things science (e.g. \emph{research},
\emph{scienc}, \emph{data}, \emph{scientist}), academia (e.g.
\emph{phd}, \emph{univers}, \emph{student}) or non-academic professions
(\emph{manag}, \emph{librarian}) are among the most common ones.
Followers also seem to frequently include words describing their
activities (e.g. \emph{learn}, \emph{develop}) or words pointing towards
the principles of Open Science (\emph{\#opensci}, \emph{access}) in
their profiles. Using a word cloud, I corroborate these findings by
mapping the 100 most common words.

\begin{Shaded}
\begin{Highlighting}[]
\CommentTok{# Plot most common words in followers' descriptions}
\NormalTok{desc_tidy }\OperatorTok
\StringTok{  }\KeywordTok{count}\NormalTok{(word_stem, }\DataTypeTok{sort =} \OtherTok{TRUE}\NormalTok{) }\OperatorTok
\StringTok{  }\KeywordTok{top_n}\NormalTok{(}\DecValTok{20}\NormalTok{, n) }\OperatorTok
\StringTok{  }\KeywordTok{ggplot}\NormalTok{(}\KeywordTok{aes}\NormalTok{(}\DataTypeTok{x =} \KeywordTok{reorder}\NormalTok{(word_stem, n), }\DataTypeTok{y =}\NormalTok{ n)) }\OperatorTok{+}
\StringTok{  }\KeywordTok{geom_bar}\NormalTok{(}\DataTypeTok{stat =} \StringTok{"identity"}\NormalTok{, }\DataTypeTok{width =} \FloatTok{0.5}\NormalTok{, }\DataTypeTok{alpha =} \FloatTok{0.8}\NormalTok{, }\DataTypeTok{color =} \StringTok{"#2c3e50"}\NormalTok{, }\DataTypeTok{fill =} \StringTok{"#2c3e50"}\NormalTok{) }\OperatorTok{+}
\StringTok{  }\KeywordTok{ylab}\NormalTok{(}\StringTok{""}\NormalTok{) }\OperatorTok{+}\StringTok{ }\KeywordTok{xlab}\NormalTok{(}\StringTok{""}\NormalTok{) }\OperatorTok{+}
\StringTok{  }\KeywordTok{ggtitle}\NormalTok{(}\DataTypeTok{label =} \StringTok{"Count of words in the Open Science MOOC-ers profile descriptions"}\NormalTok{, }\DataTypeTok{subtitle =} \StringTok{" "}\NormalTok{) }\OperatorTok{+}\StringTok{ }\KeywordTok{ylim}\NormalTok{(}\DecValTok{0}\NormalTok{, }\DecValTok{1250}\NormalTok{) }\OperatorTok{+}
\StringTok{  }\KeywordTok{coord_flip}\NormalTok{() }\OperatorTok{+}\StringTok{ }\NormalTok{viz_theme}
\end{Highlighting}
\end{Shaded}

\begin{center}\includegraphics{OpenScienceMOOC-follower-analysis_files/figure-latex/word-counts-1} \end{center}

\begin{Shaded}
\begin{Highlighting}[]
\CommentTok{# Plot word cloud}
\NormalTok{desc_tidy }\OperatorTok
\StringTok{  }\KeywordTok{count}\NormalTok{(word_stem, }\DataTypeTok{sort =} \OtherTok{TRUE}\NormalTok{) }\OperatorTok
\StringTok{  }\KeywordTok{top_n}\NormalTok{(}\DecValTok{100}\NormalTok{, n) }\OperatorTok
\StringTok{  }\KeywordTok{wordcloud2}\NormalTok{(}\DataTypeTok{color =} \StringTok{"#2c3e50"}\NormalTok{)}
\end{Highlighting}
\end{Shaded}

\hypertarget{htmlwidget-ab26be86248c276def2a}{}

The word cloud above shows that at least two well-known hashtags,
\emph{\#opensci} and \emph{\#openaccess}, are prominently featured in
the Open Science Twitter MOOC-ers profiles. Are there any other hashtags
that are also frequently mentioned?

Yes, indeed. Not surprisingly, \emph{\#openscience} comes first in this
list (guilty as charged for having this in my own profile description as
well), followed by \emph{\#openaccess} and \emph{\#scicomm}. Honorable
mentions go to multiple other hashtags referring to the various branches
of Open Science: \emph{\#opendata}, \emph{\#opensource}, \emph{\#oer} -
which is short for \emph{Open Educational Resources} - and \emph{\#oa},
denoting \emph{\#openaccess}.

On a personal and admittedly slightly judgmental note, I'm proud of the
Open Science MOOC-ers for mentioning \emph{\#rstats} more often than
\emph{\#python}. Despite using both programming languages in my work, my
heart still belongs to the former - come for the data (science), stay
for the awe-inspiRing community. Though to be fair and to potentially
burst my own bubble, there are other hashtags like \emph{\#pydata} that
Pythonistas use alternately.

\begin{Shaded}
\begin{Highlighting}[]
\CommentTok{# Plot most common hashtags in MOOC-ers profiles}
\NormalTok{desc_tidy }\OperatorTok
\StringTok{  }\KeywordTok{filter}\NormalTok{(}\KeywordTok{str_detect}\NormalTok{(word, }\StringTok{"#}\CharTok{\textbackslash{}\textbackslash{}}\StringTok{S+"}\NormalTok{)) }\OperatorTok\StringTok{ }\CommentTok{# filter hashtags}
\StringTok{  }\KeywordTok{count}\NormalTok{(word, }\DataTypeTok{sort =} \OtherTok{TRUE}\NormalTok{) }\OperatorTok
\StringTok{  }\KeywordTok{mutate}\NormalTok{(}\DataTypeTok{word =} \KeywordTok{reorder}\NormalTok{(word, n)) }\OperatorTok
\StringTok{  }\KeywordTok{top_n}\NormalTok{(}\DecValTok{20}\NormalTok{, n) }\OperatorTok
\StringTok{  }\KeywordTok{ggplot}\NormalTok{(}\KeywordTok{aes}\NormalTok{(word, n)) }\OperatorTok{+}
\StringTok{  }\KeywordTok{geom_bar}\NormalTok{(}\DataTypeTok{stat =} \StringTok{"identity"}\NormalTok{, }\DataTypeTok{width =} \FloatTok{0.5}\NormalTok{, }\DataTypeTok{alpha =} \FloatTok{0.8}\NormalTok{, }\DataTypeTok{color =} \StringTok{"#2c3e50"}\NormalTok{, }\DataTypeTok{fill =} \StringTok{"#2c3e50"}\NormalTok{) }\OperatorTok{+}
\StringTok{  }\KeywordTok{xlab}\NormalTok{(}\StringTok{""}\NormalTok{) }\OperatorTok{+}\StringTok{ }\KeywordTok{ylab}\NormalTok{(}\StringTok{""}\NormalTok{) }\OperatorTok{+}\StringTok{ }\KeywordTok{ggtitle}\NormalTok{(}\StringTok{"Most frequently used hashtags in the Open Science MOOC-ers profile descriptions"}\NormalTok{, }\DataTypeTok{subtitle =} \StringTok{" "}\NormalTok{) }\OperatorTok{+}
\StringTok{  }\KeywordTok{coord_flip}\NormalTok{() }\OperatorTok{+}\StringTok{ }\NormalTok{viz_theme }\OperatorTok{+}\StringTok{ }\KeywordTok{ylim}\NormalTok{(}\DecValTok{0}\NormalTok{, }\DecValTok{250}\NormalTok{)}
\end{Highlighting}
\end{Shaded}

\begin{center}\includegraphics{OpenScienceMOOC-follower-analysis_files/figure-latex/hashtags-1} \end{center}

In addition to the hashtags, I looked at the @-mentions in the Open
Science MOOC-ers Twitter profiles as well. Among the most common
mentions of the followers are (presumably) their respective past and
current institutional affiliations. Here, the universities Oxford,
Stanford and Maastricht were mentioned quite frequently.

Other profiles include mentions of research fellowship or online
programmes such as the
\href{https://ec.europa.eu/programmes/horizon2020/en/h2020-section/marie-sklodowska-curie-actions}{Marie
Skłodowska-Curie Actions} program of the EU (@mscactions), the
\href{https://www.stifterverband.org/freies-wissen}{Fellows Freies
Wissen} Open Science program funded by Wikimedia, the Stifterverband,
and VolkswagenStiftung (@openscifellows) as well as the
\href{https://foundation.mozilla.org/en/opportunity/mozilla-open-leaders/}{Mozilla
Open Leaders} program by Mozilla (@mozopenleaders).

Lastly, there are also mentions indicating affiliations with
institutions which devote (parts of) their efforts to Open Science,
namely \href{https://carpentries.org/}{The Carpentries}, an open global
community for teaching foundational coding and data science skills, or
\href{https://elifesciences.org/}{eLife}, an open-access journal for
research in the life and biomedical sciences.

\begin{Shaded}
\begin{Highlighting}[]
\CommentTok{# Plot most common mentions in MOOC-ers profiles}
\NormalTok{desc_tidy }\OperatorTok
\StringTok{  }\KeywordTok{filter}\NormalTok{(}\KeywordTok{str_detect}\NormalTok{(word, }\StringTok{"@}\CharTok{\textbackslash{}\textbackslash{}}\StringTok{S+"}\NormalTok{)) }\OperatorTok\StringTok{ }\CommentTok{# filter mentions}
\StringTok{  }\KeywordTok{count}\NormalTok{(word, }\DataTypeTok{sort =} \OtherTok{TRUE}\NormalTok{) }\OperatorTok
\StringTok{  }\KeywordTok{mutate}\NormalTok{(}\DataTypeTok{word =} \KeywordTok{reorder}\NormalTok{(word, n)) }\OperatorTok
\StringTok{  }\KeywordTok{top_n}\NormalTok{(}\DecValTok{20}\NormalTok{, n) }\OperatorTok
\StringTok{  }\KeywordTok{ggplot}\NormalTok{(}\KeywordTok{aes}\NormalTok{(word, n)) }\OperatorTok{+}
\StringTok{  }\KeywordTok{geom_bar}\NormalTok{(}\DataTypeTok{stat =} \StringTok{"identity"}\NormalTok{, }\DataTypeTok{width =} \FloatTok{0.5}\NormalTok{, }\DataTypeTok{alpha =} \FloatTok{0.8}\NormalTok{, }\DataTypeTok{color =} \StringTok{"#2c3e50"}\NormalTok{, }\DataTypeTok{fill =} \StringTok{"#2c3e50"}\NormalTok{) }\OperatorTok{+}
\StringTok{  }\KeywordTok{xlab}\NormalTok{(}\StringTok{""}\NormalTok{) }\OperatorTok{+}\StringTok{ }\KeywordTok{ylab}\NormalTok{(}\StringTok{""}\NormalTok{) }\OperatorTok{+}\StringTok{ }\KeywordTok{ggtitle}\NormalTok{(}\StringTok{"Most frequently used mentions in the Open Science MOOC-ers profile descriptions"}\NormalTok{, }\DataTypeTok{subtitle =} \StringTok{" "}\NormalTok{) }\OperatorTok{+}\StringTok{ }\KeywordTok{ylim}\NormalTok{(}\DecValTok{0}\NormalTok{, }\DecValTok{15}\NormalTok{) }\OperatorTok{+}
\StringTok{  }\KeywordTok{coord_flip}\NormalTok{() }\OperatorTok{+}\StringTok{ }\NormalTok{viz_theme}
\end{Highlighting}
\end{Shaded}

\begin{center}\includegraphics{OpenScienceMOOC-follower-analysis_files/figure-latex/mentions-1} \end{center}

Up to this point, the content analysis relied on words as individual
units. Since we're also interested in extracting co-occurring words or
word sequences (examples might be \emph{phd student} or \emph{data
science}), we now go one step further and analyze the relationship
between two consecutive words by tokenizing the profile texts into pairs
of adjacent words, called \emph{bigrams}.

\begin{Shaded}
\begin{Highlighting}[]
\CommentTok{# Clean descriptions (again) and get bigrams}
\NormalTok{bigrams_tidy <-}\StringTok{ }\NormalTok{data_df }\OperatorTok
\StringTok{  }\KeywordTok{mutate}\NormalTok{(}\DataTypeTok{description_clean =} \KeywordTok{gsub}\NormalTok{(}\StringTok{"^$"}\NormalTok{, }\OtherTok{NA}\NormalTok{, }\KeywordTok{trimws}\NormalTok{(description))) }\OperatorTok
\StringTok{  }\KeywordTok{mutate}\NormalTok{(}\DataTypeTok{description_clean =} \KeywordTok{gsub}\NormalTok{(}\StringTok{"http}\CharTok{\textbackslash{}\textbackslash{}}\StringTok{S+}\CharTok{\textbackslash{}\textbackslash{}}\StringTok{s*"}\NormalTok{, }\StringTok{""}\NormalTok{, description_clean)) }\OperatorTok
\StringTok{  }\KeywordTok{unnest_tokens}\NormalTok{(bigram, description_clean, }\DataTypeTok{token =} \StringTok{"ngrams"}\NormalTok{, }\DataTypeTok{n =} \DecValTok{2}\NormalTok{) }\OperatorTok
\StringTok{  }\KeywordTok{separate}\NormalTok{(bigram, }\KeywordTok{c}\NormalTok{(}\StringTok{"word1"}\NormalTok{, }\StringTok{"word2"}\NormalTok{), }\DataTypeTok{sep =} \StringTok{" "}\NormalTok{) }\OperatorTok
\StringTok{  }\KeywordTok{filter}\NormalTok{(}\OperatorTok{!}\NormalTok{word1 }\OperatorTok\StringTok{ }\NormalTok{stop_words}\OperatorTok{$}\NormalTok{word, }\OperatorTok{!}\NormalTok{word1 }\OperatorTok\StringTok{ }\NormalTok{stop_german}\OperatorTok{$}\NormalTok{word, }\OperatorTok{!}\NormalTok{word1 }\OperatorTok\StringTok{ }\NormalTok{stop_french}\OperatorTok{$}\NormalTok{word, }
         \OperatorTok{!}\NormalTok{word1 }\OperatorTok\StringTok{ }\NormalTok{stop_espanol}\OperatorTok{$}\NormalTok{word, }\OperatorTok{!}\NormalTok{word2 }\OperatorTok\StringTok{ }\NormalTok{stop_words}\OperatorTok{$}\NormalTok{word, }\OperatorTok{!}\NormalTok{word2 }\OperatorTok\StringTok{ }\NormalTok{stop_german}\OperatorTok{$}\NormalTok{word, }\OperatorTok{!}\NormalTok{word2 }\OperatorTok\StringTok{ }\NormalTok{stop_french}\OperatorTok{$}\NormalTok{word, }\OperatorTok{!}\NormalTok{word2 }\OperatorTok\StringTok{ }\NormalTok{stop_espanol}\OperatorTok{$}\NormalTok{word)}
\end{Highlighting}
\end{Shaded}

The following plot shows the bigrams that the Open Science MOOC-ers most
frequently include in their Twitter profiles.

While several bigrams seem to point to their academic and non-academic
positions (e.g. \emph{phd student/candidate}, \emph{assistant
professor}, \emph{data scientist}, \emph{associate professor},
\emph{research fellow}, \emph{project manager}), other bigrams
presumably describe the followers' professional interests, for instance
\emph{scholarly communication}, \emph{machine learning}, \emph{data
management} or \emph{digital humanities}.

\begin{Shaded}
\begin{Highlighting}[]
\CommentTok{# Plot most common bigrams}
\NormalTok{bigrams_tidy }\OperatorTok
\StringTok{  }\KeywordTok{count}\NormalTok{(word1, word2, }\DataTypeTok{sort =} \OtherTok{TRUE}\NormalTok{) }\OperatorTok
\StringTok{  }\KeywordTok{unite}\NormalTok{(bigram, }\KeywordTok{c}\NormalTok{(}\StringTok{"word1"}\NormalTok{, }\StringTok{"word2"}\NormalTok{), }\DataTypeTok{sep =} \StringTok{" "}\NormalTok{) }\OperatorTok\StringTok{ }
\StringTok{  }\KeywordTok{filter}\NormalTok{(bigram }\OperatorTok{!=}\StringTok{ "NA NA"}\NormalTok{) }\OperatorTok
\StringTok{  }\KeywordTok{top_n}\NormalTok{(}\DecValTok{20}\NormalTok{, n) }\OperatorTok
\StringTok{  }\KeywordTok{ggplot}\NormalTok{(}\KeywordTok{aes}\NormalTok{(}\DataTypeTok{x =} \KeywordTok{reorder}\NormalTok{(bigram, n), }\DataTypeTok{y =}\NormalTok{ n)) }\OperatorTok{+}
\StringTok{  }\KeywordTok{geom_bar}\NormalTok{(}\DataTypeTok{stat =} \StringTok{"identity"}\NormalTok{, }\DataTypeTok{width =} \FloatTok{0.5}\NormalTok{, }\DataTypeTok{alpha =} \FloatTok{0.8}\NormalTok{, }\DataTypeTok{color =} \StringTok{"#2c3e50"}\NormalTok{, }\DataTypeTok{fill =} \StringTok{"#2c3e50"}\NormalTok{) }\OperatorTok{+}
\StringTok{  }\KeywordTok{xlab}\NormalTok{(}\StringTok{""}\NormalTok{) }\OperatorTok{+}\StringTok{ }\KeywordTok{ylab}\NormalTok{(}\StringTok{""}\NormalTok{) }\OperatorTok{+}\StringTok{ }\KeywordTok{ggtitle}\NormalTok{(}\StringTok{"Most common bigrams in the Open Science MOOC-ers profile descriptions"}\NormalTok{, }\DataTypeTok{subtitle =} \StringTok{" "}\NormalTok{) }\OperatorTok{+}\StringTok{ }\KeywordTok{ylim}\NormalTok{(}\DecValTok{0}\NormalTok{, }\DecValTok{200}\NormalTok{) }\OperatorTok{+}
\StringTok{  }\KeywordTok{coord_flip}\NormalTok{() }\OperatorTok{+}\StringTok{ }\NormalTok{viz_theme}
\end{Highlighting}
\end{Shaded}

\begin{center}\includegraphics{OpenScienceMOOC-follower-analysis_files/figure-latex/bigram-counts-1} \end{center}

These bigram counts can also be transformed into a graph object and
visualized as a bigram network, with the nodes and edges being defined
as follows: The source is the first word and the target is the second
word in a pair of consecutive words. Edges are the connections between
two nodes whenever both words co-occurred at least ten times in the
followers' profile descriptions. This threshold was chosen for visual
reasons in order to make the depicted bigrams easier to read. The width
of the edges reflects how common or rare each bigram is.

Compared to the previous results, the bigram network allows for a deeper
insight into the interests and activities of the Open Science Twitter
MOOC-ers. As expected, common nodes in the network are \emph{research}
(often followed by an academic position), \emph{data}, \emph{science}
and \emph{phd}. Other pairs and triplets mention societal challenges
such as \emph{human rights}, \emph{mental health} or \emph{global/public
health}, indicating an increased awareness for social issues within the
community. Some followers also make clear that they tweet from personal
accounts and that the views they express are their own and retweets are
not necessarily endorsements. But see for yourself:

\begin{Shaded}
\begin{Highlighting}[]
\CommentTok{# Calculate bigram counts and transform into graph object}
\NormalTok{bigram_graph <-}\StringTok{ }\NormalTok{bigrams_tidy }\OperatorTok
\StringTok{  }\KeywordTok{count}\NormalTok{(word1, word2, }\DataTypeTok{sort =} \OtherTok{TRUE}\NormalTok{) }\OperatorTok
\StringTok{  }\KeywordTok{na.omit}\NormalTok{() }\OperatorTok
\StringTok{  }\KeywordTok{filter}\NormalTok{(n }\OperatorTok{>}\StringTok{ }\DecValTok{10}\NormalTok{) }\OperatorTok
\StringTok{  }\KeywordTok{graph_from_data_frame}\NormalTok{()}

\CommentTok{# Plot bigram network}
\KeywordTok{ggraph}\NormalTok{(bigram_graph, }\DataTypeTok{layout =} \StringTok{"nicely"}\NormalTok{) }\OperatorTok{+}
\StringTok{  }\KeywordTok{geom_edge_link}\NormalTok{(}\KeywordTok{aes}\NormalTok{(}\DataTypeTok{edge_alpha =} \FloatTok{0.1}\NormalTok{, }\DataTypeTok{width =}\NormalTok{ n), }\DataTypeTok{color =} \StringTok{"#91aac3"}\NormalTok{, }\DataTypeTok{show.legend =} \OtherTok{FALSE}\NormalTok{) }\OperatorTok{+}
\StringTok{  }\KeywordTok{geom_node_point}\NormalTok{(}\DataTypeTok{color =}  \StringTok{"#2c3e50"}\NormalTok{, }\DataTypeTok{size =} \DecValTok{5}\NormalTok{, }\DataTypeTok{alpha =} \FloatTok{0.9}\NormalTok{) }\OperatorTok{+}
\StringTok{  }\KeywordTok{geom_node_label}\NormalTok{(}\KeywordTok{aes}\NormalTok{(}\DataTypeTok{label =}\NormalTok{ name), }\DataTypeTok{vjust =} \DecValTok{1}\NormalTok{, }\DataTypeTok{hjust =} \FloatTok{0.5}\NormalTok{, }\DataTypeTok{label.size =} \FloatTok{0.05}\NormalTok{, }
                  \DataTypeTok{label.padding =} \KeywordTok{unit}\NormalTok{(}\FloatTok{0.15}\NormalTok{, }\StringTok{"lines"}\NormalTok{), }\DataTypeTok{label.r =} \FloatTok{0.05}\NormalTok{, }\DataTypeTok{fill =} \StringTok{"#ffffff66"}\NormalTok{, }
                  \DataTypeTok{color =} \StringTok{"#2c3e50"}\NormalTok{, }\DataTypeTok{repel =} \OtherTok{TRUE}\NormalTok{) }\OperatorTok{+}\StringTok{ }
\StringTok{  }\KeywordTok{ggtitle}\NormalTok{(}\StringTok{"Bigram network of the Open Science MOOC-ers profile descriptions"}\NormalTok{, }\DataTypeTok{subtitle =} \StringTok{"Edge width relative to number of times each bigram occurs"}\NormalTok{) }\OperatorTok{+}
\StringTok{  }\NormalTok{viz_theme }\OperatorTok{+}\StringTok{ }\KeywordTok{theme}\NormalTok{(}\DataTypeTok{axis.line =} \KeywordTok{element_blank}\NormalTok{(), }
                    \DataTypeTok{axis.text =} \KeywordTok{element_blank}\NormalTok{(), }
                    \DataTypeTok{axis.title =} \KeywordTok{element_blank}\NormalTok{(), }
                    \DataTypeTok{axis.ticks =} \KeywordTok{element_blank}\NormalTok{())}
\end{Highlighting}
\end{Shaded}

\begin{center}\includegraphics{OpenScienceMOOC-follower-analysis_files/figure-latex/bigram-network-1} \end{center}

The final part of this analysis deals with the sentiments conveyed in
the Open Science MOOC-ers profile descriptions. The sentiment analysis
below is based on a dictionary approach using Bing Liu et al.'s
\href{https://www.cs.uic.edu/~liub/FBS/sentiment-analysis.html}{sentiment
lexicon}. This dictionary contains approximately 6800 English sentiment
words that were classified according to whether they have a positive or
negative connotation, along with their assigned numerical values. These
lists can then be used to match the words that followers use in their
profile description with those included in the dictionary.

The following plots show, first, the total number of positive and
negative sentiments and, second, the sentiment distribution. As we can
see here, the overall sentiment of the followers' profile descriptions
is mostly positive, meaning that the texts contain more positively than
negatively connotated words. Please note that I excluded the word
\emph{endorsement} from the analysis.

\begin{Shaded}
\begin{Highlighting}[]
\CommentTok{# Remove the following terms}
\NormalTok{senti_rm <-}\StringTok{ "endorsement"}

\CommentTok{# Calculate and plot total sentiment scores (bing)}
\NormalTok{desc_tidy }\OperatorTok
\StringTok{  }\KeywordTok{filter}\NormalTok{(word }\OperatorTok\StringTok{ }\NormalTok{senti_rm) }\OperatorTok
\StringTok{  }\KeywordTok{inner_join}\NormalTok{(}\KeywordTok{get_sentiments}\NormalTok{(}\StringTok{"bing"}\NormalTok{), }\DataTypeTok{by =} \StringTok{"word"}\NormalTok{) }\OperatorTok\StringTok{  }
\StringTok{  }\KeywordTok{count}\NormalTok{(word, sentiment) }\OperatorTok
\StringTok{  }\KeywordTok{ggplot}\NormalTok{(}\KeywordTok{aes}\NormalTok{(sentiment, n)) }\OperatorTok{+}\StringTok{  }
\StringTok{  }\KeywordTok{geom_bar}\NormalTok{(}\KeywordTok{aes}\NormalTok{(}\DataTypeTok{fill =}\NormalTok{ sentiment), }\DataTypeTok{stat =} \StringTok{"identity"}\NormalTok{, }\DataTypeTok{alpha =} \FloatTok{0.8}\NormalTok{, }\DataTypeTok{width =} \FloatTok{0.5}\NormalTok{) }\OperatorTok{+}
\StringTok{  }\KeywordTok{scale_fill_manual}\NormalTok{(}\DataTypeTok{values =} \KeywordTok{c}\NormalTok{(}\StringTok{"#620000"}\NormalTok{,}\StringTok{"#006262"}\NormalTok{)) }\OperatorTok{+}\StringTok{ }
\StringTok{  }\KeywordTok{scale_x_discrete}\NormalTok{(}\DataTypeTok{labels =} \KeywordTok{c}\NormalTok{(}\StringTok{"Negative"}\NormalTok{, }\StringTok{"Positive"}\NormalTok{)) }\OperatorTok{+}
\StringTok{  }\KeywordTok{xlab}\NormalTok{(}\StringTok{""}\NormalTok{) }\OperatorTok{+}\StringTok{ }\KeywordTok{ylab}\NormalTok{(}\StringTok{""}\NormalTok{) }\OperatorTok{+}\StringTok{ }\KeywordTok{ggtitle}\NormalTok{(}\StringTok{"Count of positive and negative sentiments in the Open Science MOOC-ers profile descriptions"}\NormalTok{, }\DataTypeTok{subtitle =} \StringTok{"Sentiment lexicon by Bing Liu and collaborators"}\NormalTok{) }\OperatorTok{+}\StringTok{ }\KeywordTok{ylim}\NormalTok{(}\DecValTok{0}\NormalTok{, }\DecValTok{2000}\NormalTok{) }\OperatorTok{+}
\StringTok{  }\KeywordTok{theme}\NormalTok{(}\DataTypeTok{legend.position =} \StringTok{"none"}\NormalTok{) }\OperatorTok{+}\StringTok{ }\NormalTok{viz_theme }
\end{Highlighting}
\end{Shaded}

\begin{center}\includegraphics{OpenScienceMOOC-follower-analysis_files/figure-latex/sentiment-counts-1} \end{center}

\begin{Shaded}
\begin{Highlighting}[]
\CommentTok{# Plot sentiment distribution}
\NormalTok{desc_tidy }\OperatorTok
\StringTok{  }\KeywordTok{filter}\NormalTok{(word }\OperatorTok\StringTok{ }\NormalTok{senti_rm) }\OperatorTok
\StringTok{  }\KeywordTok{inner_join}\NormalTok{(}\KeywordTok{get_sentiments}\NormalTok{(}\StringTok{"bing"}\NormalTok{), }\DataTypeTok{by =} \StringTok{"word"}\NormalTok{) }\OperatorTok\StringTok{ }
\StringTok{  }\KeywordTok{count}\NormalTok{(screenName, word, sentiment) }\OperatorTok
\StringTok{  }\KeywordTok{group_by}\NormalTok{(screenName, sentiment) }\OperatorTok
\StringTok{  }\KeywordTok{summarise}\NormalTok{(}\DataTypeTok{sum =} \KeywordTok{sum}\NormalTok{(n)) }\OperatorTok
\StringTok{  }\KeywordTok{spread}\NormalTok{(sentiment, sum, }\DataTypeTok{fill =} \DecValTok{0}\NormalTok{) }\OperatorTok
\StringTok{  }\KeywordTok{mutate}\NormalTok{(}\DataTypeTok{difference =}\NormalTok{ positive }\OperatorTok{-}\StringTok{ }\NormalTok{negative) }\OperatorTok
\StringTok{  }\KeywordTok{ggplot}\NormalTok{(}\KeywordTok{aes}\NormalTok{(}\DataTypeTok{x =}\NormalTok{ difference)) }\OperatorTok{+}
\StringTok{  }\KeywordTok{geom_density}\NormalTok{(}\DataTypeTok{color =} \StringTok{"#2c3e50"}\NormalTok{, }\DataTypeTok{fill =} \StringTok{"#2c3e50"}\NormalTok{, }\DataTypeTok{alpha =} \FloatTok{0.8}\NormalTok{) }\OperatorTok{+}
\StringTok{  }\KeywordTok{xlab}\NormalTok{(}\StringTok{"Sentiment"}\NormalTok{) }\OperatorTok{+}\StringTok{ }\KeywordTok{ylab}\NormalTok{(}\StringTok{"Density"}\NormalTok{) }\OperatorTok{+}\StringTok{ }\KeywordTok{ggtitle}\NormalTok{(}\StringTok{"Distribution of sentiments in the Open Science MOOC-ers profile descriptions"}\NormalTok{, }\DataTypeTok{subtitle =} \StringTok{"Sentiment lexicon by Bing Liu and collaborators"}\NormalTok{) }\OperatorTok{+}\StringTok{ }\KeywordTok{ylim}\NormalTok{(}\DecValTok{0}\NormalTok{, }\FloatTok{1.6}\NormalTok{) }\OperatorTok{+}
\StringTok{  }\NormalTok{viz_theme}
\end{Highlighting}
\end{Shaded}

\begin{center}\includegraphics{OpenScienceMOOC-follower-analysis_files/figure-latex/sentiment-distribution-1} \end{center}

But which words are actually classified as positive and negative? The
following overview shows that several of the words the Open Science
MOOC-ers presumably use to describe both themselves (e.g.
\emph{enthusiast}, \emph{advocate}, \emph{lover}) and their preferences
and characteristics (e.g. \emph{love(s)}, \emph{passionate},
\emph{proud}) are among the most frequently used positive words in their
profile descriptions. Negative words include for instance the word
\emph{cancer}, which may either indicate the respective followers' own
research interests or possibly refer to their own medical history.

\begin{Shaded}
\begin{Highlighting}[]
\CommentTok{# Plot most common positive and negative words}
\NormalTok{desc_tidy }\OperatorTok\StringTok{ }
\StringTok{  }\KeywordTok{filter}\NormalTok{(word }\OperatorTok\StringTok{ }\NormalTok{senti_rm) }\OperatorTok
\StringTok{  }\KeywordTok{inner_join}\NormalTok{(}\KeywordTok{get_sentiments}\NormalTok{(}\StringTok{"bing"}\NormalTok{), }\DataTypeTok{by =} \StringTok{"word"}\NormalTok{) }\OperatorTok\StringTok{ }
\StringTok{  }\KeywordTok{count}\NormalTok{(word, sentiment, }\DataTypeTok{sort =} \OtherTok{TRUE}\NormalTok{) }\OperatorTok
\StringTok{  }\KeywordTok{group_by}\NormalTok{(sentiment) }\OperatorTok
\StringTok{  }\KeywordTok{top_n}\NormalTok{(}\DecValTok{10}\NormalTok{, n) }\OperatorTok
\StringTok{  }\KeywordTok{ungroup}\NormalTok{() }\OperatorTok
\StringTok{  }\KeywordTok{mutate}\NormalTok{(}\DataTypeTok{word =} \KeywordTok{reorder}\NormalTok{(word, n)) }\OperatorTok
\StringTok{  }\KeywordTok{ggplot}\NormalTok{(}\KeywordTok{aes}\NormalTok{(word, n)) }\OperatorTok{+}
\StringTok{  }\KeywordTok{geom_bar}\NormalTok{(}\DataTypeTok{stat =} \StringTok{'identity'}\NormalTok{, }\KeywordTok{aes}\NormalTok{(}\DataTypeTok{fill =}\NormalTok{ sentiment), }\DataTypeTok{width =} \FloatTok{0.5}\NormalTok{, }\DataTypeTok{alpha =} \FloatTok{0.8}\NormalTok{) }\OperatorTok{+}
\StringTok{  }\KeywordTok{scale_fill_manual}\NormalTok{(}\DataTypeTok{name =} \StringTok{"Sentiment"}\NormalTok{, }
                    \DataTypeTok{labels =} \KeywordTok{c}\NormalTok{(}\StringTok{"Negative"}\NormalTok{, }\StringTok{"Positive"}\NormalTok{), }
                    \DataTypeTok{values =} \KeywordTok{c}\NormalTok{(}\StringTok{"negative"}\NormalTok{ =}\StringTok{ "#620000"}\NormalTok{, }\StringTok{"positive"}\NormalTok{ =}\StringTok{ "#006262"}\NormalTok{)) }\OperatorTok{+}\StringTok{ }
\StringTok{  }\KeywordTok{xlab}\NormalTok{(}\StringTok{""}\NormalTok{) }\OperatorTok{+}\StringTok{ }\KeywordTok{ylab}\NormalTok{(}\StringTok{""}\NormalTok{) }\OperatorTok{+}\StringTok{ }\KeywordTok{ggtitle}\NormalTok{(}\StringTok{"Most common positive and negative words in the Open Science MOOC-ers profile descriptions"}\NormalTok{, }\DataTypeTok{subtitle =} \StringTok{"Sentiment lexicon by Bing Liu and collaborators"}\NormalTok{) }\OperatorTok{+}
\StringTok{  }\NormalTok{viz_theme }\OperatorTok{+}\StringTok{ }\KeywordTok{coord_flip}\NormalTok{() }\OperatorTok{+}\StringTok{ }\KeywordTok{ylim}\NormalTok{(}\DecValTok{0}\NormalTok{, }\DecValTok{150}\NormalTok{)}
\end{Highlighting}
\end{Shaded}

\begin{center}\includegraphics{OpenScienceMOOC-follower-analysis_files/figure-latex/pos-neg-word-counts-1} \end{center}

To get a broader picture of the most common positive and negative words,
they can also be visualize in a comparison cloud.

\begin{Shaded}
\begin{Highlighting}[]
\CommentTok{# Plot comparison cloud}
\NormalTok{desc_tidy }\OperatorTok
\StringTok{  }\KeywordTok{inner_join}\NormalTok{(}\KeywordTok{get_sentiments}\NormalTok{(}\StringTok{"bing"}\NormalTok{), }\DataTypeTok{by =} \StringTok{"word"}\NormalTok{) }\OperatorTok\StringTok{ }
\StringTok{  }\KeywordTok{count}\NormalTok{(word, sentiment, }\DataTypeTok{sort =} \OtherTok{TRUE}\NormalTok{) }\OperatorTok
\StringTok{  }\KeywordTok{acast}\NormalTok{(word }\OperatorTok{~}\StringTok{ }\NormalTok{sentiment, }\DataTypeTok{value.var =} \StringTok{"n"}\NormalTok{, }\DataTypeTok{fill =} \DecValTok{0}\NormalTok{) }\OperatorTok
\StringTok{  }\KeywordTok{comparison.cloud}\NormalTok{(}\DataTypeTok{colors =} \KeywordTok{c}\NormalTok{(}\StringTok{"#620000"}\NormalTok{, }\StringTok{"#006262"}\NormalTok{),}
                   \DataTypeTok{max.words =} \DecValTok{100}\NormalTok{)}
\end{Highlighting}
\end{Shaded}

\begin{center}\includegraphics{OpenScienceMOOC-follower-analysis_files/figure-latex/comparison-cloud-1} \end{center}

\subsection{Conclusion}\label{conclusion}

So what does this analysis tell us about the Open Science MOOC Twitter
community? The findings demonstrate that a substantial number of
followers of the official account are based in larger cities across
Western Europe and North America. Hence, there is some evidence for the
Open Science Twitter bubble being geographically clustered. Still, the
Open Science MOOC-ers seem to be a diverse and inclusive community
overall as both genders are represented almost equally and most
followers tweet from unverified personal accounts. When it comes to the
professional activities of the community members, they tend to either be
PhD students, hold more senior academic positions or seem to work in
closely related fields such as data science or science communication.
Moreover, several followers are officially affiliated with universities,
others with national and international Open Science fellowship programs
or organization promoting Open Science. Many followers also actively
spread the values and principles of Open Science by including hashtags
referring to the various branches of Open Science - including Open
Access, Open Data, Open Source, and Open Educational Resources - in
their profile descriptions, which mostly convey positive sentiments.

Taken together, the empirical results strongly indicate that the Open
Science Twitter MOOC-ers are indeed a bunch of enthusiastic, diverse,
and highly-educated science afficionados who dedicate their time and
effort to actively advocate and advance Open Science.


\end{document}
